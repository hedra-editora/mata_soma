\begin{itemize}
\item \textbf{Introdução à \versal{SOMA}: terapia e pedagogia anarquista do corpo} trata de um processo terapêutico realizado em grupo, corporal, e que busca no pensamento anarquista uma crítica às mais variadas forma de poder impregnadas no comportamento individual e nas relações sociais.
O grupo de terapia funciona como um micro"-laboratório social, no qual desenvolvemos uma análise libertária do comportamento de cada um a partir da relação junto ao outro. Daí vem sua originalidade: terapia como criação e afirmação de si, em que a construção das práticas de liberdade é o antídoto para combater os conflitos gerados pelas relações sociais hierarquizadas.
O presente livro é uma ``porta de entrada'' a esta técnica terapêutica libertária. Aqui encontram"-se os principais conceitos, as bases teóricas e a metodologia da Soma. Em linguagem simples e direta, deve despertar o leitor para os caminhos singulares e insurgentes em direção à construção de práticas alegres, afirmativas e guerreiras.
  
\item \textbf{João da Mata} tem 50 anos, é somaterapeuta, Psicólogo (\versal{CRP}. 29962/05), Mestre em Filosofia e Doutor em Psicologia (\versal{UFF}) e Doutor em Sociologia Econômica e das Organizações (Univ. de Lisboa -- Portugal). Pós"-Doutor em História (\versal{UFF}).
Seu primeiro contato com Roberto Freire e a Soma aconteceu em 1988, em Recife -- \versal{PE}, desde então vem pesquisando"-a e divulgando"-a. Em todos estes anos, coordenou grupos de terapia em várias cidades brasileiras
(Recife, Fortaleza, Salvador, Rio de Janeiro, São Paulo, Curitiba, Florianópolis, Porto Alegre etc.) e na Europa (Lisboa, Barcelona e Madrid).
Publicou em 1993, junto com Freire, o livro \emph{\versal{SOMA} -- Vol. \versal{III} -- Corpo a Corpo} (Ed. Guanabara Koogan). Em 2001 lançou \emph{A Liberdade do Corpo} (Ed. Imaginário). Em 2006 publicou junto com Freire \emph{O Tesão pela Vida} (Ed. Francis), e em 2007, o livro \emph{Prazer e Rebeldia -- o materialismo hedonista de Michel Onfray} (Ed. Achiamé).
Além dos grupos da Soma, desenvolve atendimento individual utilizando"-se da técnica somática.

\end{itemize}

