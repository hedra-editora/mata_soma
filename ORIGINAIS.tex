%Introdução à

%SOMA

%terapia e pedagogia anarquista do corpo

%João da Mata

\chapter*{}
\thispagestyle{empty}

\vspace*{\fill}

\begin{flushright}
\begin{adjustwidth}{2.0cm}{}
\raggedleft\scriptsize\emph{A \versal{\emph{Coleção Ataque}} irrompe sob efeito de junho de 2013.
Esse acontecimento recente da história das lutas sociais no Brasil, a um só
tempo, ecoa combates passados e lança novas dimensões para os
enfrentamentos presentes. O critério zero da coleção é o choque com os
poderes ocorrido durante as \emph{jornadas de
junho}, mas não só. Busca"-se captar ao menos uma pequena parte do fluxo de
radicalidade (anti)política que escorre pelo planeta a despeito da
tristeza cívica ordenada no discurso da esquerda institucionalizada. Um
contrafluxo ao que se convencionou chamar de onda conservadora. Os
textos reunidos são, nesse sentido,
anárquicos, mas não apenas de autores e temas ligados aos
anarquismos. Versam sobre batalhas de
rua, grupos de enfrentamento das forças policiais, demolição da forma"-prisão que
ultrapassa os limites da prisão"-prédio. Trazem também análises sobre os
modos de controle social e sobre o terror do racismo de Estado. Enfim, temas de enfrentamento com
escritas que possuem um alvo.}

\emph{O nome da coleção foi tomado de um antigo
selo punk de São Paulo que, em 1985, lançou a coletânea \emph{Ataque
Sonoro}. Na capa do disco dois mísseis, um soviético e outro
estadunidense, apontam para a cidade de São Paulo, uma metrópole do que
ainda se chamava de terceiro mundo. Um anúncio, feito ao estilo audaz
dos punks, do que estava em jogo: as forças rivais atuam juntas contra o
que não é governado por uma delas. Se a configuração mudou de lá para
cá, a lógica e os alvos seguem os mesmos. Diante das mediações e
identidades políticas, os textos desta coleção optam pela tática do
ataque frontal, conjurando as falsas dicotomias que organizam a
estratégia da ordem. Livros curtos para serem levados no bolso, na
mochila ou na bolsa, como pedras ou coquetéis molotov.
Pensamento"-tática que anima o enfrentamento colado à urgência do
presente. Ao serem lançados, não se espera desses livros mais do que
efeitos de antipoder, como a beleza de exibições pirotécnicas. Não há
ordem, programa, receita ou estratégia a serem seguidos. Ao atacar
radicalmente a única esperança possível é que se perca o controle e,
como isso, dançar com o caos dentro de si. Que as leituras produzam
efeitos no seu corpo.}

\medskip

\emph{Acácio Augusto \& Renato Rezende}
\end{adjustwidth}
\end{flushright}

\pagebreak
\thispagestyle{empty}

\begin{vplace}[30]
\begin{flushright}
Para os participantes dos grupos da Soma.

São eles que nos ensinam e ajudam

na criação de nossas heterotopias.
\end{flushright}
\end{vplace}

%\textbf{ORELHA}

%Cartola, em parceria com Elton Medeiros, canta em \emph{O sol nascerá}:
%A sorrir / Eu pretendo levar a vida / Pois chorando / Eu vi a mocidade
%perdida... Talvez sem perceber, Cartola expressava em versos uma potente
%atitude ética, um verdadeiro modo de vida alegre e afirmativo.

%Penso ser a Soma - uma terapia anarquista, a aposta em algo semelhante.
%Transformar nossa existência em um grande atelier a serviço de uma obra
%de arte, no qual o propósito em criar um sentido artístico e belo à vida
%é também uma aposta ética e afirmativa. Acredito ser este o maior
%objetivo terapêutico na Soma: inventar, continuamente, a existência como
%potência.

%A Somaterapia ou apensa Soma é um processo terapêutico-pedagógico
%realizado em grupo, com ênfase na articulação entre o trabalho corporal
%e o uso da linguagem verbal e que buscou no pensamento e na ação
%libertária tanto uma crítica às mais variadas forma de poder impregnadas
%no comportamento individual como também nas relações sociais.

%O grupo de terapia funciona como um micro-laboratório social, no qual
%desenvolvemos uma analítica libertária do comportamento de cada um a
%partir da relação junto ao outro. A Soma tem duração pré-definida e não
%costuma passar de um ano e meio, com encontros periódicos (são quatro
%sessões mensais). Sua maior originalidade vem daí: terapia como criação
%e afirmação de si, em que a construção das práticas de liberdade é o
%antídoto para combater a neurose gerada pelas relações sociais
%hierarquizadas.

%Criada no Brasil pelo escritor e médico Roberto Freire, em plena
%ditadura civil-militar nos anos de 1960, os primeiros grupos de Soma
%funcionaram de forma clandestina para atender aos ativistas políticos.
%Não havia uma terapia em que os jovens perseguidos pelos militares
%pudessem confiar, pois a paranóia de denuncia era enorme.

%Assim, surgia uma terapia fortemente ligada às lutas por liberdade,
%contra a normalidade perversa, instituída e instaurava no Brasil.
%Passados mais de quarenta anos de sua criação, a Soma permanece atenta
%aos mecanismos de controle e captura das liberdades individuais e suas
%repercussões sobre a vida emocional das pessoas.

%O livro que você tem em mãos é uma ``porta de entrada'' a esta técnica
%terapêutica libertária. Aqui você encontrará os principais conceitos, as
%bases teóricas e a metodologia da Soma. Em linguagem simples e direta,
%deve desperta-lo para os caminhos singulares e insurgentes em direção à
%construção desta vida que Cartola canta em samba: alegre, afirmativa e
%guerreira.

%Rio de Janeiro, abril de 2017.

\chapter*{Apresentação\\ \emph{\emph{Soma} para potências de liberdades}}
\addcontentsline{toc}{chapter}{Apresentação, \footnotesize{\emph{por Edson Passetti}}}

\begin{flushright}
\emph{Edson Passetti}\footnote{Professor Livre"-Docente de Ciências Políticas da \versal{PUC"-SP}.}
\end{flushright}

\emph{Anarquismo somático}. E que venham anarquias e muitos mais
anarquismos singulares, firmes, preciosos e sintonizados com a
atualidade. Como veio Roberto Freire, um homem especial e comum,
libertário encenador, jornalista, escritor, guerreiro, médico, terapeuta
e inventor de \emph{soma}.

\emph{Soma, uma terapia anarquista}: escultura de si para a vida livre.
\emph{Soma}: uma vivência lúdico"-socializante de libertação diante de
situações opressivas. \emph{Soma} é uma reviravolta em si, em grupo e
com o somaterapeuta em breve tempo e de maneira autogestionária. É uma
experimentação repleta de novidades, viagens, danças, lutas e amizades.
É tudo isso e um pouco mais, apresentado de maneira generosa por João da
Mata.

O leitor propício à \emph{somaterapia}, um libertário, um estudante de
psicologia, um ativista da antipsiquiatria, um acadêmico, um jovem ou
adulto, interessados em liberdade encontrarão neste consistente livro
uma exposição sobre a história da \emph{Soma}, seus fundamentos,
práticas e ressonâncias na atualidade.

O platonismo, o cristianismo e o cartesianismo, dentre tantos
\emph{ismos} que afirmam condutas, não encontram repercussões em
\emph{Soma}. Aqui, não há corpo apartado da razão, mas corpo
indissociável da razão surpreendendo e desestabilizando a moral e as
extensões corporais no trabalho, no conhecimento, na existência. Corpo
não como lugar de normalizações, espaço das domesticações das emoções e
da adaptação para a ordem. Apenas \emph{soma}: espaço de libertação de
opressões, de bloqueios e de modelos. E como tal \emph{soma} é antes de
tudo um risco para cada um, o grupo e o somaterapeuta. \emph{Soma}
potencializa liberdades e dissolve o poder do terapeuta em função do
prazer e da afirmação das singularidades. \emph{Soma} relações
horizontais desestabilizando estruturas.

Na \emph{Soma} cada pessoa é ``única''. \emph{Soma} é a revolta contra o
``indivíduo massificado'', as hierarquias, as sujeições, o poder do
superior que domina, disciplina e controla. É também um modo de
rompimento com o assujeitamento, essa prática eficaz de disseminar o
amor à obediência, ao conformismo, à domesticada participação
institucional, aos duplos"-vínculos.

\emph{Soma} só pode acontecer em grupo, pois são as pessoas associadas
sem hierarquias que afirmam sua vida única. Elas provocam a
possibilidade de convivência com as diferenças na igualdade, o que é
próprio dos amigos. Pela \emph{Soma} se rompe com as neuroses
disseminadas pela sociedade. João da Mata sublinha, incisivamente, que a
\emph{Soma} é para ser vivida em grupo e de maneira autogestionária para
que cada um defenda ``sua unicidade, autorregulação espontânea e
originalidade única''.

Com a \emph{Soma} avançamos sobre o fluxo das normalizações e provocamos
a força de uma ``política do cotidiano''. Questionamos os autoritarismos
não pelos excessos ou perversões do soberano, mas a verticalidade das
relações de poder. Abrimo"-nos para a liberdade no acasalamento, na
família, no trabalho, na nossa existência. Não tememos investir contra a
ordem.

\emph{Soma} atiça nosso instinto de irritabilidade em função de uma
``agressividade como potência'', diferenciada dos animais e do que se
convencionou chamar de Homem, essa representação oblíqua da existência
de cada um reduzida a um universal que nos dispõe às variadas ideologias
dos reformadores, restauradores e de certos revolucionários.

\emph{Soma} começou como luta contra a ditadura militar. Sua instauração
está relacionada a pensadores que procuraram contestar a ordem e afirmar
pessoalidades por meio de suas experiências. Compuseram, pela delicadeza
de Roberto Freire, um grupo de referência para a fundamentação de suas
propostas voltadas para liberar e equilibrar energias de um corpo
assolado por neuroses.

A \emph{somaterapia} atua no cotidiano, no ``aqui"-agora''. Não é
tratamento nem administração de neuroses, mas uma maneira de lidar com a
percepção nas situações de vida. O somaterapeuta anda com Wilhelm Reich,
a beleza do sexo e liberação de couraças. Conversa com Fritz Perls
incorporando suas contribuições relativas a como equacionar as situações
de vida que abrimos e que precisam ser concluídas. Alerta para o fato
que podemos conter o aparecimento da neurose fortalecendo em cada um a
sua capacidade de decisão. De David Cooper veio a imprescindível
contestação à psiquiatria e à noção de doença mental. Na companhia de
Gregory Bateson aprendemos a problematizar a noção de \emph{cura}, pois
a relação doença"-cura"-doença não está no horizonte dos praticantes de
\emph{soma}. Ao contrário, \emph{Soma} libera cada único das possíveis
prisões pelos ``duplos"-vínculos''.

A \emph{Soma} existe pela formação de um grupo de pensadores cujas
práticas não dissociam corpo e razão como Roberto Freire e os jovens
somaterapeutas, Wilhem Reich, Fritz Perls, David Cooper, George Bateson
e mais recentemente o libertário filósofo francês Michel Onfray.
\emph{Soma} é para mudar a vida e não para adaptar qualquer indivíduo à
sociedade. \emph{Soma} é própria do somaterapeuta e seu grupo de
pensadores, do grupo de pessoas livres que iniciam a \emph{Soma,} deste
grupo com o somaterapeuta e de cada único nas associações que inventará
posteriormente. \emph{Soma} não é propriedade de ninguém!

\emph{Soma} é para quem gosta de gente, para quem aprecia lutas de
libertação, resistências, maneiras de dar forma à liberdade. Anda com
pensadores sobre o corpo e também incorpora práticas de libertação de
escravos como a capoeira angola. \emph{Soma} é pensamento e prática
heterodoxos não confinados a um espaço exclusivo. Está na sala, no
galpão, no terreiro e se desloca com o bando nômade por espaços abertos
e amplos, culminando, depois de palavras, gestos, silêncios, danças e
lutas, num cara a cara de cada único com o somaterapeuta e com os demais
nômades. \emph{Soma} é um jeito de existir único, rebelde e inventor de
outras maneiras de realizar a utopia da existência livre. De lutar no
presente para se livrar do capitalismo de hoje e dos socialismos
ultrapassados.

\emph{Soma} é um jeito anarquista pra quem quiser! Entre pelo livro de
João da Mata. Habite a \emph{Soma} e desabitue"-se de normalizações. Se
isso \emph{te} levar às práticas anarquistas ficaremos mais fortes, mas
se a sua passagem \emph{te} der e você nos der um pouco mais de
liberdade, também estaremos satisfeitos por compartilharmos a
\emph{sinceridade} dos livres.

\chapter*{Conceitos básicos da Soma}
\addcontentsline{toc}{chapter}{I. Conceitos básicos da Soma}

\section{Definição da palavra \versal{SOMA}}

A Somaterapia traz em seu nome a inscrição mais clara e direta de sua
proposta terapêutica: uma terapia feita a partir do corpo em sua
totalidade. A palavra Soma foi adotada por Roberto Freire ao buscar uma
expressão que ultrapassasse as noções de psiquismo e racionalidade. No
início de suas pesquisas, ainda no começo dos anos de 1970, as investigações de
Freire vieram a constituir"-se como uma espécie de
\emph{antipsicoterapia}, embasadas nas descobertas reichianas e nas
práticas libertárias. Apenas em 1976 seu trabalho passa a ser nomeado de
Somaterapia, quando então procura valorizar a importância do corpo
enquanto totalidade somática.

A palavra \emph{soma} vem do grego e significa \emph{corpo}. Porém,
quando nos referimos ao corpo, não estamos meramente falando do que está
abaixo do pescoço, como é comum pensar. Mas à totalidade do ser,
respeitando um princípio unicista, segundo o qual o importante é o
todo somático, sem divisões entre corpo e mente ou mundo sensível e
mundo das ideias, por exemplo. Nos interessa compreender o funcionamento
da vida a partir do conjunto dessa vida, sem hierarquia de alguma função
vital e emocional sobre quaisquer outras.

Assim, na Soma, trabalhamos com a noção do ser humano em seu conjunto
funcional e dinâmico do ser, no qual o indivíduo é uma unidade
indivisível. Esta compreensão da existência humana busca desenvolver uma
visão de totalidade, superando práticas e teorias que favoreçam a
fragmentação e alienação do indivíduo, sem levar em conta as condições
sociais, históricas, de gênero e de raça e tantas outras que incidem na
produção de nossa subjetividade

Qualquer divisão funcional, utilizada no tratamento dos conflitos e
impasses humanos, tende a desprezar ou minimizar a forma como cada um
sente e percebe sua inscrição no mundo e a maneira como lida nele. Elas
podem servir apenas para a compreensão didática da vida, mas nunca para
avaliar a complexidade e dinâmica da existência humana. Isolar uma parte
do todo para efeito de estudo de algum distúrbio leva a um diagnóstico
limitado e insuficiente.

Isto explica por que eliminamos o prefixo ``psico'' do nosso trabalho
terapêutico. Não entendemos que a Soma seja uma psicoterapia, no sentido
usual que este termo é utilizado. Isto porque não considerarmos ser
a mente o foco de ação terapêutica, e sim todo o soma, toda a pessoa,
com a mente inclusive, mas fazendo parte do todo somático. A Somaterapia
se constitui desta forma como um processo no qual o pensamento e a razão
são utilizados a serviço do entendimento da vida, porém nunca de maneira
exclusiva ou prioritária. Procuramos valorizar também \emph{como} a
pessoa sente e percebe a vida, quais os fatores que o emocionam, que
produzem mais prazer ou satisfação, dor ou tristeza, por exemplo.
Enfim, procuramos entender a vida de forma mais ampla que meramente pela
via da compreensão racional.

O pensamento unicista foi combatido de várias maneiras e formas durante
o percurso histórico da humanidade. A ideia da separação entre mente e
corpo, ou melhor dizendo, entre mundo sensível e mundo das ideias,
remete à tradição platônica. Para o platonismo, o corpo e a alma estão
separados, e mais, são incompatíveis um com o outro. O corpo é visto
como um cárcere e consequentemente um empecilho para a prática do
pensamento. É no corpo que nascem e desenvolvem"-se os desejos e
prazeres, os temores e as imaginações, todas elas sensações fúteis que,
para o platonismo, impedem o acesso à verdade. O que importa é a alma, e
o comando que ela exerce sobre o corpo. Para isto, é preciso desprezar a
matéria, pois ela é vista como uma prisão que impedirá que se cuide das
virtudes da alma.

Platão, em \emph{Fédon}, sugere que se esqueça do corpo e se busque, da
melhor maneira possível, ocupar"-se dos cuidados da alma para poder
examinar os objetos em si apenas com ela. Segundo o filósofo, o corpo
exige uma demanda de cuidados por demais custosos, entre os quais a
alimentação, o vestuário, o cuidado com a saúde, entre tantos, além de
sentimentos como o medo, a tristeza, o desejo, enfim, uma demanda de
sensações que atrapalham a alma a ascender ao verdadeiro conhecimento.
Através do corpo, não é possível conhecermos puramente nada. O que é
preciso fazer, para o platonismo, é afastar"-se das sensações
corporais e aproximar"-se estritamente do conhecimento, que %pois este
fornecerá informações precisas e verdadeiras sobre a vida.

Esta noção foi muito bem aproveitada e estendida à cultura moderna pelo
cristianismo, que acreditava ser o espírito mais importante que a
matéria. O platonismo estabelece uma irredutível oposição entre corpo e
alma, que no cristianismo se dará sob a forma de carne e espírito. Tal
dualidade trouxe consequências marcantes à civilização ocidental, para a qual,
de um lado, a alma apresenta"-se como modelo e, do outro, o corpo como
exemplo negativo a ser evitado. O inteligível, o espiritual e a visão
idealista que despreza o sensível, o corpo, a carne e o prazer, passam a
constituir uma moral dominante que segue firme até os dias de hoje. Sob
o ponto de vista psicológico, além de limitar a compreensão e o
entendimento da existência, produz sérios bloqueios sensoriais, sensuais
e afetivos.

Mais tarde, foram os racionalistas que vincularam a existência humana ao
pensamento. Descartes criou uma das mais famosas afirmações da
filosofia, através do conhecido aforismo ``Penso, logo existo'', que
aparece pela primeira vez em \emph{O discurso do método} (1637). A
afirmação sugere que o pensamento e a consciência do pensar são os
verdadeiros substratos da existência. E mais: sua função está
hierarquicamente acima e é mais importante que o corpo.

O racionalismo está presente em boa parte da nossa cultura. A visão
racional do homem enfoca a mente como a parte de maior valor na
experiência humana. O corpo, de importância secundária, serviria apenas
de sustentação para a mente, o local privilegiado da existência. A
separação abissal que se anuncia a partir daí produz um corpo
esquizofrênico, cindido em duas partes que supostamente funcionam com o
pensamento e suas operações mais refinadas de um lado; e o corpo e seu
funcionamento biológico e primário de outro.

Na Soma, o uso da razão e consequentemente o entendimento racional sobre
a vida e seus conflitos abrange uma dimensão fundamental do processo
terapêutico. No entanto, buscamos também a compreensão de tais impasses
através das emoções, das sensações, dos prazeres, enfim, de um conjunto
de informações que advém da corporeidade, de uma sabedoria que o corpo
possui e que nos auxilia a perceber a vida além do pensamento. Seria
algo próximo do que disse, certa vez, o poeta Fernando Pessoa, em %QUEBRAR OS VERSOS DIREITO
\emph{O guardador de rebanhos}: ``E os meus pensamentos são todos
sensações. Penso com os olhos e com os ouvidos. E com as mãos e os pés.
E com o nariz e a boca''. Entendemos que os pensamentos são afetados e
afetam sensações e emoções; da mesma forma que o jeito como sentimos
algo, como nos emocionamos por determinados estímulos estéticos, éticos
ou sociais, incidem sobre a forma como processamos ideias e pensamentos.

Assim, entendemos que os sentimentos exercem forte influência sobre a
razão. Isso não significa dizer que a razão tenha menor importância que
as emoções ou que haja alguma forma de mensurar a presença de cada uma
na condução da vida. Reconhecer e saber agenciar sentimentos e emoções,
em suas complexas redes de funcionamento e influências socioculturais,
requer de cada um perceber"-se como um organismo que possui
funcionamentos integrados e se encontra permanentemente interativo com o
meio ambiente físico e social.

Para a Soma, portanto, tal compreensão do ser humano --- seu corpo,
emoções, pensamentos, cultura, manifestações sociais e história de vida
--- compõem um quadro unicista. Toda concepção unicista está amparada na
ideia de que cada indivíduo é um universo em si próprio. Uma experiência
ampla e complexa e ao mesmo tempo completa. Esta maneira de entender a
vida humana faz com que a Somaterapia se ocupe em desenvolver seu
processo levando em consideração o todo da existência de cada pessoa:
seu corpo (pensamentos, emoções, sensações etc.) e suas extensões
corporais (trabalho, conhecimento, vida social etc.).

\section{A auto-regulação organísmica}

Outro conceito fundamental ao entendimento e objetivos da Soma diz
respeito à autorregulação. Uma pessoa que vive a sua unicidade dispõe de
um mecanismo próprio e singular que lhe auxilia a determinar sua
existência no mundo. Este é o \emph{princípio da autorregulação
espontânea,} uma capacidade que todos temos de gerir e regular nossas
vidas. Defendemos que, fora o necessário aprendizado dos costumes
sociais e culturais que se vive num determinado momento, cada pessoa
detém capacidades próprias em criar e adequar suas funções vitais ao que
necessita e deseja, e o faz de forma satisfatória se não for
bloqueada ou ensinada a agir de determinadas maneiras a partir de
referenciais colocados como corretos ou normais.

Isto nos dá uma condição prática e despretensiosa em relação à terapia.
Para isso, é preciso localizar o bloqueio, o que bloqueia a vida de uma
pessoa, como ela está bloqueada, e ajudá"-la a se desbloquear. Quando ela
conseguir isto, suas funções vitais tendem a se reorganizarem e operarem
a partir de ações próprias. Estas funções vitais dizem respeito aos
mecanismos básicos da vida, tais como a vivência do amor, da
sexualidade, da criatividade, da convivência social e uma série de
outros campos da existência.

A autorregulação é uma condição psicológica e política necessária a uma
vida autônoma. É a capacidade que temos de governar nossa própria vida,
direcionar nossa existência segundo nossas criações, nas rotas e nos
caminhos que cada um constrói de acordo com suas experiências, demandas
e possibilidades. Autorregular nossa vida é saber respeitar nossos
desejos e demandas, articulando"-os com o meio em que estamos inseridos,
através de um cálculo sempre novo entre o \emph{eu} e o \emph{outro}. O
prazer torna"-se nosso principal referencial ético na elaboração de uma
``escultura de si'', onde cada pessoa constrói sua existência dando a
forma e os contornos que deseja.

Este conceito, tal como entendemos, surgiu dos estudos de Fritz Perls
(1893-1970), criador da Gestatl"-terapia. Ele sugeriu que cada pessoa tem
a capacidade de reconhecer sua ``autorregulação organísmica'', ou seja,
definir sua existência no mundo a partir do que o seu organismo sinaliza
como mais ou menos importante a cada momento de sua vida. Assim, cada
ser se autorregula conforme a necessidade do próprio organismo, no aqui
e agora. Esta noção de organismo é bem mais do que está abaixo de nossa
pele, mas diz respeito ao conjunto de fatores que nos constitui como
sujeitos: nossa trajetória de vida, questões políticas e econômicas, de
gênero e raciais, históricas e culturais etc., enfim, a uma séria de
atravessamentos que agiram e continuam agindo na permanente produção de
nossa singularidade e aos processos de subjetivação por cada um
vivenciado.

Então, poderíamos afirmar que uma pessoa autorregulada dá à sua vida
ritmo e forma de acordo com as suas necessidades e prioridades. Ela ama
as pessoas que mais lhe agradam, trabalha naquilo que sente maior
aptidão e competência, convive com as pessoas que mais se identifica. A
autorregulação espontânea é determinada pela busca do prazer, que
representa um importante referencial, auxiliando cada um na capacidade
de guiar sua própria existência.

Uma das características que podemos sinalizar no estado neurótico é o
distanciamento da autorregulação em direção à heterorregulação (do grego
\emph{hetero,} que significa diferente, irregular). Uma situação
neurótica muitas vezes está diretamente relacionada a um mecanismo
heterorregulador, impossibilitando a pessoa de conduzir sua vida a
partir de suas próprias demandas e necessidades. Ao contrário, quando
não consegue identificá"-las, tende a ``copiar'' modelos de
comportamentos, condutas apreendidas nas relações sociais ou mesmo em
padrões estereotipados de proceder na vida, muitas vezes repetidos ou
automatizados. Gasta"-se muita energia vital neste tipo de representação,
além, claro, de ser inútil.

Muitas das dificuldades apresentadas em terapia dizem respeito à
incapacidade ou inabilidade que algumas pessoas têm para definir o que
querem e gostam, em eleger seus campos de desejos. Um conflito
neurótico emerge, por exemplo, quando a pessoa vive a representação de
papéis e personagens que correspondem às expectativas alheias, distantes
de suas próprias referências. Daí a sensação de incompetência e
impotência geral, sobretudo amorosa e criativa, decorrente da falta de
energia desperdiçada pela heterorregulação, quase sempre atrelada à
necessidade de agradar ou atender as demandas de outrem.

Adotando referenciais externos, a vida de uma pessoa heterorregulada
distancia"-se da realização do seu próprio prazer. A heterorregulação é
consequência direta das práticas que regem a sociedade em que estamos
inseridos, geralmente autoritária e massificante, que governa condutas e
oferece padrões de comportamento, sutil ou explicitamente, aos
indivíduos que a compõe. Isto acontece sob a forma de valores morais, de
leis e normas que surgem de várias fontes: da mídia, do Estado, das
religiões, do mercado, das escolas e principalmente da molécula básica
da sociedade --- a familiar tradicional, geralmente chantagista e
controladora.

Ainda na infância, convivemos com relações que misturam amor e carinho
com chantagens e autoritarismos, limitando e condicionando nossos
gostos, preferências e escolhas. O medo de perder o amor e a proteção
dos pais torna a criança vulnerável e sem condições de reagir, a não ser
conformando"-se e submetendo"-se. Esta educação é reforçada de forma geral nas
escolas, que em sua grande maioria apoia"-se em práticas disciplinadoras,
fortalecendo a apatia e diminuição do poder crítico das pessoas.
Educadas em ambientes autoritários, a maioria das pessoas acabam por
chegar à vida adulta dependentes e inseguras, o que só faz aumentar a
sensação de heterorregulação diante de suas relações sociais. Nas
sociedades autoritárias e hierarquizadas, o poder e o controle
substituem o prazer e a autonomia.

A presença cotidiana da autoridade não deve ser confundida com o jeito
rude de falar ou com o tom agressivo na voz, ainda que estes possam ser
componentes eventuais do autoritarismo. Nos lares, nas escolas e em
outros espaços da sociedade, muitas vezes o autoritarismo vem dosado em
tom suave, até meigo, e sempre muito bem intencionado, para o ``nosso
bem''. No exercício de governar a vida dos outros, os mecanismos de
autoritarismo, ontem e hoje, se transvestem em roupagens das mais
diversas, que vão do pai repressor e violento até aquele que ouve e dá
``bons conselhos''.

Assim, a heterorregulação é determinada pelo exercício do autoritarismo
e da submissão, implícitos ou explícitos. Procuramos, no decorrer do
trabalho da Soma, auxiliar cada indivíduo na construção criativa de sua
autonomia, passando pelo reconhecimento e pelo exercício cotidiano de
sua autorregulação, assim como pelo enfrentamento dos mecanismos
heterorreguladores.

Ao estabelecer uma análise crítica do comportamento individual e social
dos membros do grupo, buscamos perceber como ocorreram ou continuam a
ocorrer os mecanismos pelos quais as pessoas acabam por delegar poder a
outras, seja por preocupação excessiva com a opinião alheia, seja por
colocarem"-se no lugar de assujeitadas. O que nos interessa como processo
terapêutico libertário é favorecer meios e instrumentos com os quais as
pessoas possam exercer suas práticas de liberdade, com capacidade de
entendimento e decisão, e que rejeitem a servidão voluntária.

\section{A originalidade única do ser}

Como temos até aqui mostrado, a Soma enquanto terapia libertária busca
estimular as pessoas à vivência de sua autorregulação, e para tanto,
enfrentar os mecanismos de poder que bloqueiam sua autonomia. Tal
objetivo visa combater a existência de comportamentos que são oferecidos
como modelos nos diferentes espaços da malha social, levando as pessoas
ao exercício da diferença na condução de sua existência. Este caminho
propicia o encontro e a prática do que há de único nas pessoas, o que
chamamos de \emph{originalidade única do ser}.

Este deveria ser o objetivo de qualquer terapia: facilitar o indivíduo
ao exercício de sua singularidade, rompendo, para isso, as amarras que
foram impregnadas ao longo de sua formação, desde aquelas surgidas a
partir de práticas repressivas ou mesmo outras sugeridas como adequadas
e normais. No entanto, o que observamos em alguns casos são práticas
terapêuticas adaptadoras, sem nenhuma função transformadora e
revolucionária. Quando a psicologia opera nestes termos, distante de
uma análise política dos conflitos emocionais, torna"-se meramente um
agente a serviço das práticas de poder, auxiliando a adaptação das
pessoas à sociedade tal qual ela é e funciona, sem estabelecer uma
análise crítica dos mecanismos normatizadores presentes nas relações
humanas. Entendemos que um processo terapêutico deve ocupar"-se de
perspectivas que incidam em uma direção diametralmente oposta: fomentar
a luta incessante pelo exercício do que há de mais singular na vida de
uma pessoa, mesmo vivendo em uma sociedade hierarquizada e
massificadora.

O que existe em alguém e o diferencia dos outros é o resultado de
diferentes acontecimentos que ocorrem ao longo de sua vida. Os processos
de subjetivação derivam da permanente conexão entre fatores sociais,
culturais, urbanísticos, econômicos e ético"-políticos, que se inscrevem
em um determinado tempo histórico na vida da cada um de nós. A noção de
singularidade vista como identidade cede lugar ao múltiplo, em
permanente devir para atualizar"-se nas modulações de si e em seu
entorno. Podemos ser semelhantes em diversas características, tanto na
morfologia e fisiologia, quanto no comportamento. Porém, existe algo
singular, próprio de cada ser, que nunca houve, não há e nem haverá
nunca em outra pessoa.

A ideia da originalidade única não é algo estanque, rígido ou mesmo um
ponto a ser atingido. Também não se trata da noção de essência, algo
pronto ou \emph{a priori} que existe em alguém. Ela é dinâmica e está em
constante transformação e construção. O que se define como singular num
determinado momento de vida pode se transformar em algo novo,
dependendo das afetações internas e externas, mudanças de rumos de vida,
enfim, do próprio movimento da existência.

Nesta perspectiva, nosso referencial de saúde aponta para a vivência da
originalidade única, para o exercício da autorregulação e pela busca da
unicidade. Resultado de uma cartografia pessoal, este processo é criado
por cada um, através de caminhos sinuosos e inéditos, que desembocam em
uma estética da existência. Há um sentido de beleza cada vez que nos
construímos por meio de uma elaboração de si e criamos práticas livres
em diferentes espaços de nossas vidas. Ao mesmo tempo, estas práticas
criam linhas de fuga a um social que não cessa em querer moldar
individualidades.

Toda vez que alguém não consegue expressar sua singularidade como ser,
acaba reproduzindo padrões e personagens sociais aprendidos. Umas das
características da neurose é exatamente o rompimento, causado por
bloqueios externos, da unicidade, o que torna a pessoa dividida, fraca,
insegura e impotente. Isto acontece na medida em que tais bloqueios
impedem o livre funcionamento da autorregulação espontânea do nosso
organismo. Nestas condições, o exercício da diferença fica bastante
comprometido.

Isso explica, ao menos em parte, a relação adotada por nós entre
psicologia e política. Indivíduos presos aos padrões de comportamento de
um determinado conjunto de valores identitários são peças fundamentais
para que relações autoritárias se desenvolvam e se mantenham, pois se
tornam facilmente manipuláveis. As relações de disciplina e controle
perpetuam"-se na existência de pessoas submissas, não"-criativas e
conformadas.

Este conceito de neurose explica por que a Soma apresenta"-se como um
trabalho de natureza política. Não a política institucional, mas a
política do cotidiano como veremos mais adiante, presente em nossas
relações diárias e mais próximas. Nosso objetivo é revelar esta política
e descobrir as relações e práticas de poder explícitas ou disfarçadas,
seus efeitos e suas consequências. A terapia se faz na medida em que a
pessoa produz as mudanças para sair do círculo vicioso oprimido"-opressor
para uma relação política entre iguais na criação de seu prazer e
liberdade.

Para chegar a isso, a Soma só poderia ser uma terapia de grupo. Como a
neurose emerge na relação social, é aí que ela mais claramente se
mostra. Na Soma, procuramos criar nos grupos de terapia a noção de um
laboratório microssocial onde, através da dinâmica de grupo
autogestionária, formam"-se relações suficientemente variadas para
mostrar as dificuldades de cada um na defesa de sua unicidade,
autorregulação espontânea e originalidade única. Mais adiante,
mostraremos na metodologia da Soma e no desenvolvimento de seu processo
terapêutico como ocorre este trabalho.

Construir uma vida potente passa pelo redimensionamento de nossos modos
de existência frente aos valores que são oferecidos nos tempos atuais,
marcados pela intensificação de um individualismo vulgar, dominado pelo
capitalismo hegemônico e globalizado. A ética capitalista e sua
ideologia de mercado acabam por se tornar responsáveis em boa parte pela
produção de subjetividade que vivemos, invadindo nossos corpos, condutas
e inter"-relações. Sua presença não está apenas no cifrão da conta
corrente, mas se insere de forma capilar nas atitudes cotidianas que
estabelecemos conosco e com o mundo. É fundamental que o processo
terapêutico esteja atento a isso, ampliando nosso entendimento e ação
diante desses processos que vão além da dimensão psicológica.

\section{A política do cotidiano}

A Soma é uma terapia em grupo que funciona como um micro"-laboratório
social, possibilitando a percepção do que cada um é a partir de suas
relações com os demais membros do grupo. O processo terapêutico se dá
pelas descobertas individuais, auxiliado pela forma como somos
percebidos pelos outros. Este ``espelho social'' que representa o
trabalho em grupo da Soma nos auxilia a reconhecer como agimos e lidamos
com os outros.

O campo social de estudo que representa o grupo, no qual seus membros
interagem com os demais e onde a terapia se desenvolve, nos possibilita
trabalharmos no que chamamos de uma \emph{política do cotidiano}. Para
nós, é importante percebermos como lidamos com as malhas de poder em
nossa volta, quais as práticas políticas e éticas que desenvolvemos e
como somos afetados por elas. E especialmente como reproduzimos --- muitas
vezes sem perceber --- os jogos de poder nas relações humanas, repetindo
um mesmo circuito muitas vezes indesejado, mas que não damos conta de
mudar. Esta associação entre psicologia e política, presente no processo
terapêutico da Soma, nos possibilita entender, primeiro, a articulação
entre o comportamento humano e os jogos de poder, e segundo, como se dá
a implicação de cada um neste processo. O grupo se torna, então, o campo
privilegiado onde a terapia se processa, por meio de uma dinâmica de
grupo autogestiva que proporciona sociabilidades mais horizontais,
combatendo hierarquias e mecanismos de dominação.

Compreendemos o ato político como sendo uma contribuição cultural de
nossa formação no desenvolvimento antropológico da espécie humana,
complementando a ação espontânea, biológica, necessária à sua mais
completa e satisfatória organização social. A atividade política serve
basicamente para solucionar conflitos pessoais nos projetos de
organização social. Mas, historicamente o autoritarismo humano vem
impedindo o incremento da solidariedade e as decisões em consenso na sua
ação política, impedindo o exercício de uma justiça social verdadeira.
No estudo da política do cotidiano, o que nos interessa é analisar a
implicação sobre a subjetividade das pessoas, decorrente das práticas de
poder disseminadas na malha social, presentes no dia a dia. Estes
conflitos de poder no campo microssocial são o germe inicial das práticas
de poder na esfera mais ampla da sociedade e, portanto, um campo
privilegiado de estudo.

É na política do cotidiano que questionamos o autoritarismo que permeia
o conjunto das relações sociais. A família, as relações afetivas, a
escola, os meios de comunicação de massa, os partidos políticos que
acabam presos às articulações políticas do Estado, o centralismo
democrático e a despolitização das relações sociais são identificados
como alicerces do autoritarismo. Neste sentido, não nos interessam as
rígidas análises que compõem divisão entre esquerda e direita: ambas
acabam acreditando na ordem imposta pelo Estado. Mesmo as esquerdas
acabam optando pela via autoritária para a libertação coletiva. Ao
contrário, propomos a política revolucionária do cotidiano baseada na
auto"-organização e nas práticas de liberdade associadas ao prazer. Fazer
política revolucionária é algo que se dá em todas as áreas da vida: no
amor, no trabalho, nas organizações sociais etc.

A terapia acontece neste campo de estudo, onde cada membro é estimulado
a perceber como age nesta esfera da micropolítica como reflexo das
ações e práticas que exercemos no cotidiano. O grupo representa, assim,
um espaço privilegiado para o desenvolvimento da terapia individual em
cada um, tendo para tanto o espelho crítico do outro e a oportunidade
para a transformação terapêutica, que quase sempre estão atreladas aos
redimensionamentos das práticas de sociabilidades e políticas.

\section{A relação agressividade x violência}

Por fim, para concluir a definição destes conceitos básicos da Soma, vou
tentar exprimir, mesmo que sinteticamente, outra noção fundamental ao
entendimento de como vemos o surgimento e a manutenção da neurose. Estou
falando da utilização do conceito do instinto de irritabilidade animal,
definido pela biologia, como sendo o caminho para explicar as
manifestações da agressividade e da afetividade humana. A conduta
animal, desde um ser unicelular como a ameba até um ser pluricelular com
um mamífero de grande porte, passando pelo ser humano em seu meio
ambiente e em suas relações com animais da mesma e de outras espécies,
se realiza movida pelo instinto de irritabilidade. Graças a ele, o
animal foge, luta ou destrói o que o ameaça em seu meio ambiente
(através do lado agressivo de seu instinto). Ao mesmo tempo, ele se
sente atraído (pelo lado afetivo do mesmo instinto) e procura, toca e se
relaciona com o que lhe provoca prazer, basicamente para a sua
alimentação, prazer ou reprodução.

O comportamento humano em seu meio social deriva assim do instinto
biológico básico. Ele funciona em todos os seres vivos a partir deste
duplo modo de ação, por meio das sensações de prazer que seus contatos
no meio animal e cultural lhe proporcionam. A antropologia admite estas
duas instâncias na formação do comportamento humano: o biológico,
primitivo e que se transmite pela hereditariedade; o histórico,
posterior e transmitido pela evolução da cultura. Assim, o instinto da
irritabilidade animal funciona na espécie humana por meio da preservação
e da busca do prazer; pela afetividade e pela agressividade; e faz com
que o homem, em seu meio, aproxime"-se e incorpore ou se afaste e
enfrente aquilo que o agride ou de que necessita para sua sobrevivência
física e social, para o afeto e a procriação.

O desenvolvimento da cultura humana nos permitiu conhecer a busca do
prazer, com a qual coexiste e convive, em seu meio físico e social, como
a afetividade que lhe garante a sobrevivência e a possibilidade de
associação, de afeto e de procriação com membros da mesma espécie. Por
outro lado, o instinto de irritabilidade também origina a agressividade,
que defende o homem na vida animal e social, que o faz lutar para
garantir sua sobrevivência. Assim como auxilia na realização de seus
projetos de vida culturais, afetivos, criativos e produtivos. A
agressividade, quando equilibrada e bem dosada, é a principal
responsável pela ação que operamos decorrente da opção que realizamos
cotidianamente.

A afetividade e a agressividade são impulsos necessários à vida
animal e cultural dos homens. Se não forem exercidos naturalmente e de
modo satisfatório em seu equilíbrio dinâmico, o ser humano terá
dificuldades de exercer sua liberdade. Ele pode sobreviver fisicamente,
mas culturalmente estará morto. Estas reflexões procuram explicar o
comportamento animal e cultural do ser humano, e servem para
caracterizar os efeitos da apatia gerada pela diminuição da liberdade na
vida das pessoas, impedindo"-as de exercerem suas necessidades de
afetividade e agressividade nos planos biológico e animal, social e
cultural.

A agressividade é instaurada no ser humano desde a vida intra"-uterina e
depois na amamentação. A criança faz com a boca, os braços e os olhos
movimentos espontâneos de aproximação da mãe para sentir seu contato. Há
também na criança um movimento mais específico que consiste em empurrar
a cabeça para frente e abrir os lábios a fim de encontrar o seio e o
mamilo. Este é um movimento rítmico, vibrante e agressivo para a fonte
do estímulo. Os movimentos da boca, se frustrados, transformam"-se em
impulsos agressivos de morder e gritar, podendo inclusive se manifestar
de modo violento.

No ser humano, a violência é um ato compulsivo, e resulta exercício da
agressividade de forma não adequada e necessária, quando se acumulam
necessidades frustradas. A violência, então, é fruto do mau uso da
agressividade. Toda vez que abrimos mão de nossas necessidades ou nos
submetemos à situação de autoritarismo de forma implícita ou explícita,
estamos reprimindo nossa agressividade espontânea, que mais tarde se
transforma em atitudes de compulsão violenta. Assim, a agressividade
reprimida ou mal dosada pode tornar"-se um ato violento.

Sabemos ser imprescindível para uma pessoa alcançar a saúde que ela
descubra e viva sua originalidade única como ser singular, tendo que
lutar para isso contra as variadas relações hierárquicas e autoritárias
que muitas vezes se estabelecem nas famílias, escolas e sociedade como
um todo. Não a alcançando, por diversas razões, comumente torna"-se
fraco, infeliz, improdutivo, não amoroso, não criativo como a média das
pessoas. Enfim, uma pessoa com conflitos neuróticos, com sérios sintomas
de impotência e incompetência na realização de sua vida. É importante
buscar, portanto, o exercício da originalidade única e defendê"-la
através da vivência permanente de seu prazer (exercício da afetividade)
e lutar contra os mecanismos que tentam lhe opor a isso (exercício da
agressividade). Assim, o processo de construção da liberdade e da
autonomia se dá através da luta e do enfrentamento cotidiano.

Para colocar essa ideologia do prazer na política do cotidiano, em
oposição à ideologia do sacrifício estimulada pelas religiões e Estados
autoritários, utilizamos uma palavra que simboliza bem isso. Costumamos
utilizar a palavra \emph{tesão} como uma espécie de seta indicadora
desses caminhos, nos auxiliando ao exercício de nossas paixões
libertárias. Viver o nosso tesão, essa mistura de alegria, beleza e
prazer no cotidiano, nos auxilia no exercício prático dos fatores e das
coisas que fazem parte de nossa espontaneidade e originalidade única.

Nosso objetivo fundamental no processo pedagógico"-terapêu- tico que
realizamos na Soma é trabalhar sobre o tema da liberdade e da autonomia
para que cada um possa viver sua originalidade única e lutar contra os
mecanismos e forças que agem contra nossa potência. A construção das
práticas de liberdade é essencial para poder garantir a autoregulação em
nossa existência, e para que a pessoa tenha possibilidade de viver o seu
prazer. Assim, agressividade e afetividade são como os dois lados da
mesma moeda, indissociáveis e em permanente busca de equilíbrio. Deles,
resulta uma dinâmica que buscamos encontrar no exercício de uma vida
libertária no aqui"-e"-agora.

Estas reflexões a respeito dos temas até aqui apresentado buscam ampliar
a conceituação de temas importantes ao nosso trabalho e servir de base
ao entendimento teórico e metodológico da Soma. Vamos agora apresentar
as principais linhas teóricas que dão sustentação ao pensamento e ação
da Soma e depois abordaremos a prática do processo terapêutico.

\chapter*{As bases teóricas da Soma}
\addcontentsline{toc}{chapter}{II. As bases teóricas da Soma}

\epigraph{``O amor, o trabalho e o conhecimento são as fontes de nossa vida.
Deveriam também governá-la.''}{Wilhelm Reich}

A Soma é um trabalho terapêutico"-pedagógico que busca o
redimensionamento do comportamento individual junto com a construção de
sociabilidades mais livres e menos hierarquizadas. Ela surgiu da união
singular de teorias e práticas contemporâneas em alguns campos do saber,
notadamente na psicologia, na sociologia, na política e na filosofia.
Sua epistemologia e metodologia interdisciplinar lhe conferem um caráter
original e em constante movimento. Neste sentido, podemos afirmar que a
Soma é uma obra aberta, e pensada como algo inacabado, em processo de
permanente construção.

Esta característica lhe permite receber e contribuir com as descobertas
e pesquisas mais recentes no campo das ciências humanas. Foi assim desde
sua emergência e permanece até hoje através de quem a produz. A Soma não
é, portanto, uma obra acabada. Nem pretende ser. Ao contrário, buscamos
atualizar sua prática para atender às mudanças constantes que passamos
enquanto indivíduos e coletividades.

A Soma se constituiu inicialmente a partir das pesquisas e experiências
em teatro, sobre o desbloqueio da criatividade para atores, realizadas
no Centro de Estudos Macunaíma em São Paulo, Brasil, no início da década
de 1970. Através de exercícios teatrais, jogos lúdicos e debates
éticos"-políticos, Roberto Freire e uma equipe de colaboradores criaram
uma série de vivências que possibilitavam uma rica descoberta sobre o
comportamento, e suas diferentes e singulares características.

Perceber como cada um reage a diferentes situações comuns no cotidiano
das relações humanas, tais como a agressividade, a comunicação, ou a
criatividade, e a articulação com os sentimentos e emoções, permite um
entendimento daquilo que caracteriza as pessoas enquanto singularidade.
Neste processo, o que nos interessa ainda é criar novas sociabilidades,
onde a massificação ceda espaço à diferença.

A Soma atua assim, claramente na construção de espaços de liberdade, na
busca da autonomia e na produção autogestiva vividas no presente. As
influências teóricas e o momento político vivido no Brasil devido à
ditadura civil"-militar durante o surgimento da Soma encontraram uma
convergência comum na elaboração de uma terapia com objetivos
explicitamente libertários.

Neste período, os jovens que lutavam contra a ditadura não dispunham de
um método terapêutico em que pudessem confiar, politicamente, no
atendimento dos desequilíbrios emocionais e psicológicos provocados em
suas vidas pela rejeição e repressão autoritárias das famílias
burguesas, ligadas à truculência dos militares e políticos fascistas. O
medo da denúncia era tão presente que pais entregavam seus filhos,
amigos e namorados faziam o mesmo com seus companheiros. Foi neste
cenário que surgiu a Soma, fruto das experiências do Roberto Freire
em teatro, na ação política contra a ditadura e no encontro com as obras
de Wilhelm Reich, da Gestalt"-terapia e da Antipsiquiatria, tudo isso
combinado com a crítica libertária presente nos anarquismos.

Mesmo diante de uma sociedade dita democrática, após a ``abertura
política'' com o fim da ditadura civil"-militar no Brasil, vivemos num
mundo cada vez mais marcado pelos sutis mecanismos de controle. Se no
passado a presença do autoritarismo era explícita, hoje o poder navega
por camadas menos óbvias de captura das individualidades, tornando"-se
mais complexo e perverso. As correntes teóricas que veremos a seguir
estão diretamente ligadas a estas análises sociais e políticas na
atualidade e adotadas pela Soma com o propósito de criar resistências
contra as práticas de poder que tentam aniquilar o que há de singular em
cada um.

\section{A psicologia política e corporal de Wilhelm Reich}

Consideramos a obra do austríaco e ex"-psicanalista Wilhelm Reich
(1897-1957) a principal referência teórica em nosso trabalho. Como uma
espécie de espinha dorsal, a obra de Reich vai se articulando com as
demais vertentes e pensamentos teóricos que adotamos, compondo um
conjunto coerente e uno para a Soma. Foi através da obra de Reich que a
Soma comprovou as origens sociais e políticas da neurose. Nascido no
final do século \versal{XIX}, Reich foi responsável por uma das maiores
transformações da psicologia contemporânea, capaz de levar a compreensão
da neurose para além da psicologia, articulando"-a com a sociologia e a
política.

Wilhelm Reich foi um médico que logo cedo ingressou na recém criada
Sociedade Psicanalítica Internacional, quando Sigmund Freud ainda
desenvolvia os primeiros conceitos da Psicanálise e pretendia difundi"-la
e divulgá"-la para o mundo. Mais tarde, se tornou também um de seus
maiores críticos e dissidentes. De origem camponesa, Reich desde cedo
interessou"-se pelo estudo social do homem, sendo um ativo militante nas
várias atividades em que participou.

Vivia"-se na Europa no início do século \versal{XX} um forte clima de
transformações, sobretudo pela crescente importância das ideias
socialistas que eclodiam em revoluções, como a de 1917 na Rússia, e que
fatalmente influenciaram os vários segmentos científicos e culturais da
época. A diferença na compreensão dos acontecimentos sociais e políticos
vividos neste período, e a crença em suas implicações sobre os
indivíduos e a sociedade, acabaram por gerar a separação de Reich do
trabalho com Freud e sua Psicanálise.

As divergências de Reich que o levaram a ser expulso da Sociedade de
Psicanálise foram quanto às supostas causas e aos procedimentos para se
tratar a neurose. Reich passou a acreditar fortemente que um dos
principais fatores desencadeadores dos conflitos emocionais decorre de
mecanismos sociais e políticos. Para ele, a neurose forma"-se na relação
entre o indivíduo e suas sociabilidades, por meio de práticas de poder
presentes no meio social autoritário e hierarquizado, que direciona a
conduta das pessoas, desde cedo, através de padrões de comportamento que
são disseminados como verdades ainda na infância e na adolescência.

Este processo que surge em diferentes pontos da malha social ocorre por
um conjunto de regras e normas, na maioria das vezes de forma sutil e
dissimulada, mas que vai gerando um perfeito ajustamento e diminuição
de poder crítico nas pessoas. Reich afirmava categoricamente que,
enquanto existir qualquer espécie de normatização moral, social ou
política limitando a autonomia das pessoas, não se poderá falar em
liberdade real nem muito menos em saúde emocional.

Além de ser produto da heterorregulação social, a neurose se instala em
todo o corpo e não apenas na mente como acreditava a Psicanálise. Com
isso, Reich inaugura na psicologia uma nova e importante vertente
clínica, na qual o corpo passa a ser utilizado como diagnóstico (através
da leitura corporal) e local da ação terapêutica (através dos exercícios
corporais). Seu trabalho tornou"-se uma das bases do que mais tarde
passou a ser denominado de psicossomática, cujo objetivo é estabelecer
as relações entre os desequilíbrios emocionais e as doenças físicas.
Podemos apontar a psicossomática como campo de saber --- de caráter
transdisciplinar --- que integra diversas práticas para estudar os efeitos
de fatores sociais e psicológicos sobre processos orgânicos do corpo e
vice"-versa. Defende ainda que não existe separação entre mente e corpo,
que transitam nos contextos sociais, familiares, profissionais e
relacionais.

Voltemos ao Reich. Ele afirmou ser um desequilíbrio energético uma
importante característica da neurose. Mas não apenas uma energia
psíquica, ou um processo exclusivo da mente, e sim a energia única que
circula por todo o corpo. Esta energia passou a ser designada de modos
diferentes, mas com o mesmo significado: bioenergia, energia orgânica ou
energia vital.

A distribuição defeituosa e imprópria da bioenergia, principalmente na
musculatura voluntária, leva à formação do que ele veio a chamar de
\emph{Couraça Caracteriológica} ou \emph{Couraça muscular do caráter}.
Ele redescobria, assim, no fim da década de 1920 no Ocidente, o
conceito orientalista sobre energia vital (presente na acupuntura, por
exemplo), cujos escritos ele desconhecia, apesar de existentes há mais
de cinco mil anos. São sete as regiões no corpo em que se formam os
segmentos ou anéis da couraça: a região do olhos, da boca, da cervical,
dos ombros, do diafragma, do abdome e da pélvis. Cada pessoa vai
desenvolver tensionamentos em uma ou mais dessas regiões, de acordo com
sua história de vida, os processos de subjetivação e as respostas e
postura encontradas como forma de responder aos mecanismos sociais e
políticos envolvidos.

Reich observou que emoções e pensamentos têm sempre equivalentes físicos
e vice"-versa. Uma emoção traz consigo mudanças na circulação da
bioenergia relacionadas a contrações musculares localizadas e
modificações na respiração e na postura física. Isto acontece, por
exemplo, numa situação de medo: o corpo sofre diferentes alterações
fisiológicas (aumento do batimento cardíaco, dilatação das pupilas,
contrações musculares etc.) necessárias para uma melhor resposta à
ameaça. Todas estas mudanças na fisiologia corporal tendem a passar e o
corpo voltar ao estado de equilíbrio assim que a ameaça se desfazer.
Este é um mecanismo espontâneo de adaptação do ser humano a diferentes
estímulos.

Mas, se um indivíduo for submetido desde a infância a contínuas
situações de medo ou insegurança, por exemplo, as transformações
fisiológicas naturais, em vez de circunstanciais, tendem a se
cristalizar e se tornar crônica de forma inconsciente. Aquilo que
deveria acontecer em situações específicas torna"-se contínuo na vida da
pessoa, criando"-se, então, uma estrutura muscular e postural
característica que determina o seu jeito de estar no mundo: o seu
\emph{caráter}. O que Wilhelm Reich chama de caráter diz respeito à
estrutura caracteriológica de funcionamento emocional de uma pessoa, ou
seja, às características e padrões que desenvolvemos ao longo da vida.
Segundo ele, existe uma íntima relação entre posturas e gestos corporais
com aspectos do comportamento emocional de cada pessoa.

A formação da couraça neuromuscular do caráter ocorre, sobretudo,
durante nosso desenvolvimento infantil, devido a bloqueios permanentes
da sexualidade e ameaças constantes à nossa vida emocional. Ela pode se
manifestar tanto em contrações como em flacidez musculares crônicas.
São, por exemplo, regiões corporais constante e intensamente rígidas,
como ombros permanentemente arqueados e tensos ou testas franzidas. Pode
se apresentar ainda por uma respiração mais ``curta'' ou movimentos
corporais rígidos e pouco espontâneos.

Isso leva a um consumo excessivo de nossa energia vital, além de impedir
o seu circuito livre para outras áreas. Essa couraça serve, na infância,
de escudo protetor a ameaças emocionais recebidas. Na idade adulta,
limitando a relação com o mundo, a couraça impossibilita a expressão
direta da espontaneidade, da emotividade e da afetividade. Com isso, ela
passa a ser geradora dos sintomas neuróticos como fobias, angústia,
depressão, ansiedade, incompetências e impotências criativas, sexuais e
afetivas.

Estão relacionadas às doenças psicossomáticas, tais como distúrbios
sexuais, inflamações no estômago, garganta, psoríases etc. Podem ainda
aparecer de formas mais brandas, tais como garganta seca, ``nó'' na
barriga, tremores, dores de cabaça, e uma série de outras somatizações.
Para Reich, a couraça neuromuscular do caráter é a expressão física da
neurose. É a materialização corporal dos traços de comportamento e
atitudes emocionais do indivíduo, tornando"-se uma espécie de
corporificação do inconsciente.

Em observações clínicas, Reich conseguiu descobrir um dispositivo
espontâneo, disponível ao ser humano, capaz de dissolver essa armadura
defensiva e restituir, assim, o livre fluxo de energia vital. Esse
estudo deu origem ao livro, revolucionário em sua época, chamado a
\emph{Função do Orgasmo} (1927). Segundo ele, o orgasmo sexual pleno e
satisfatório, além de proporcionar grande prazer e bem"-estar, tem uma
segunda função capaz de agir sobre a couraça, dissolvendo"-a por alguns
instantes através de uma forte descarga, que afrouxa a musculatura tensa
e restitui a harmonia energética.

Além de proporcionar um dos maiores prazeres humanos, o orgasmo tem a
função de ser um agente regulador da nossa saúde. Mas, muitas vezes, a
couraça neuromuscular é tão rígida que bloqueia a realização de orgasmos
plenos, impedindo a descarga energética natural. Assim, fecha"-se um
ciclo vicioso: não se tem orgasmo, a couraça não é descronificada,
aumentam os sintomas neuróticos e isto dificulta a obtenção de orgasmos.
E mais: a pessoa encouraçada nem mesmo consegue parceiros satisfatórios
ou tem sua vida afetiva bloqueada, impedindo uma entrega mais profunda
ao outro, o que dificulta ainda mais a sua vida sexual.

O orgasmo pleno e satisfatório, capaz de agir desbloqueando a couraça, é
muito diferente que apenas a ejaculação no homem ou o prazer vaginal da
mulher. Ele se caracteriza por uma dissolução circunstancial do ego, o
que provoca movimentos e sons involuntários, assim como uma variação
momentânea da relação espaço"-tempo. Isto leva as pessoas a desfrutarem
de um profundo relaxamento e uma sensação de plenitude após o orgasmo.
Tal sensação é fruto do prazer em si, associado ao poderoso movimento %em si, não em sim
de energia vital que circula pelo corpo.

Reich comprovou que, para se tratar a neurose, era necessária uma
terapia que eliminasse a tensão crônica da couraça, restituindo a
liberdade do indivíduo de se relacionar diretamente com o mundo, sem
medos e sem escudos de defesa. Pesquisou, então, técnicas que quebrassem
este ciclo, permitindo que a circulação bioenergética fosse restituída.
Chegou então à descoberta dos \emph{equivalentes orgásticos}, ou seja,
outros meios de se chegar aos benefícios bioenergéticos proporcionados
pelo orgasmo sexual sem, contudo, estarem circunscritos ao sexo. Existem
equivalentes ditos naturais (dançar, gargalhar, chorar, bocejar,
espreguiçar etc.) e outros artificiais (massagens, exercícios), todos
com propriedades capazes de mobilizar a couraça.

Na Soma, trabalhamos com exercícios corporais próprios, outros oriundos
de jogos teatrais, danças e brincadeiras, além da capoeira angola como
veremos mais adiante. São trabalhos corporais variados que atuam sobre a
couraça neuromuscular, buscando dissolver suas tensões crônicas,
liberando a energia vital que estava sendo desperdiçada e tornando"-a
disponível para o ato de viver. Os exercícios bioenergéticos, além de
produzirem uma boa circulação energética, também possibilitam a
percepção corporal, contato de si próprio, e também a identificação
dos bloqueios. O resultado obtido através de um eficiente trabalho
corporal tanto é percebido no momento seguinte a ser concluído, como na
reserva energética que vai acumulando"-se no corpo das pessoas.

No entanto, sabemos que para eliminar os efeitos da neurose não basta
acabar com os sintomas resultantes da couraça, e sim, combater suas
causas sociais e políticas. Sintomas como angústia e depressão, por
exemplo, são \emph{sinais de alarme}, indicando que algo de errado está
acontecendo na vida da pessoa. Apagar esta ``luz vermelha'' seria como
eliminar uma febre sem combater a infecção; os sintomas apenas sinalizam
algo. É importante pesquisar e compreender os motivos que levam a vida
emocional a despertar tais sinais. Esta busca nos leva à descoberta de
relações de poder que reprimem, disciplinam e bloqueiam a vida. Não
podendo ser ela mesma, nem manifestar seus ideais e desejos, a pessoa
entra, defensivamente, em estado neurótico. A couraça neuromuscular do
caráter é a manifestação física e a consequência direta desse jogo
pedagógico autoritário.

Ao articular o estudo do comportamento humano aos mecanismos sociais e
políticos presentes na socialização, temos desenvolvido uma importante
pesquisa que vem contribuindo para o desenvolvimento e aperfeiçoamento
dos estudos sobre as causas da neurose. A principal originalidade desta
pesquisa é o estudo do comportamento ideológico das pessoas que antecede
e determina o surgimento da couraça neuromuscular. Chamamos de
\emph{couraça somática} ao conflito político e ideológico primários,
fonte dos distúrbios emocionais e psicológicos que produzem a couraça
caracteriológica. Voltar"-se ao estudo de como cada pessoa define e
conduz seu modo"-de"-vida é tão importante para nós quanto perceber como
ela demonstra seus bloqueios e sintomas.

A compreensão da couraça somática é prioritária à couraça muscular, pois
ela é a verdadeira causa da neurose. Para tanto, damos ênfase ao estudo
da política do cotidiano, buscando perceber as bases e os mecanismos dos
conflitos de poder que nela ocorrem. Sabemos, inclusive, que a couraça
neuromuscular em si não é prejudicial. Ao contrário, ela é um mecanismo
espontâneo de defesa à disposição do ser humano. O que torna a couraça
nociva e patológica é o fato de tornar"-se crônica, mantendo um estado
permanente de defesa e o consequente gasto de energia. Mas isso é
determinado pela couraça somática, fruto de um comportamento social
autoritário externo ou assumido internamente, que gera medo,
acorvadamento e passividade diante dos enfretamentos da vida.

Na Soma, o combate à couraça somática e neuromuscular é feito,
respectivamente, através do trabalho de entendimento e problematizações
em torno dos mecanismos de poder que atuaram e continuam atuando na vida
das pessoas; como também pelo restabelecimento do fluxo livre da
bioenergia, favorecendo um vitalismo e uma grande saúde. Isto se dá
através de trabalhos corporais liberadores, que funcionam como
exercícios bioenergéticos, dissolvendo a rigidez muscular da couraça. Ao
mesmo tempo, o trabalho de entendimento ético e político é feito
durante a dinâmica de grupo autogestiva no decorrer da terapia.

\section{O aqui-e-agora da Gestalt-terapia}

Outra importante escola da psicologia contemporânea para a Soma é a
Gestalt"-terapia, responsável pelo desenvolvimento prático do processo
terapêutico. Nosso trabalho é voltado para o estudo do cotidiano, para o
aqui"-e"-agora, valorizando com isso a percepção dos conflitos que estão
presentes na atualidade. Não que o passado seja descartado, mas a ênfase
é dada aos mecanismos e aspectos atuais da vida da pessoa. Desta forma,
as contribuições da Gestalt são fundamentais no processo terapêutico do
grupo, pois ela inaugurou uma nova maneira de se trabalhar o conflito
emocional a partir das observações cotidianas.

A palavra \emph{gestalt} não tem tradução direta para o português. É uma
palavra alemã que, aproximadamente, significa ``a forma como as
situações apresentam"-se e organizam"-se diante e dentro de nós''. A
Gestalt"-terapia surge na segunda metade do século \versal{XX}, através das
pesquisas de seu criador, o alemão Friedrich Perls (1893-1970), com a
cooperação de Laura Perls e Paul Goodman. Colaborador de Freud e posteriormente cliente e amigo de Reich
por algum tempo, Perls tornou"-se psicanalista e exerceu a psicanálise
durante muitos anos. De origem judaica, precisou fugir da Alemanha
nazista e se instalar na África do Sul, onde montou a base da Sociedade
Psicanalítica daquele país.

Mas foi depois de mudar para os Estados Unidos e romper com a
psicanálise que ocorreram as pesquisas decisivas que o levaram à criação
da terapia gestáltica. Ela está relacionada com a psicologia da gestalt do final do século \versal{XIX}, mas não é a mesma coisa, uma vez que combina abordagens fenomenológicas, existencialistas, dialógicas e de campo ao processo de transformação e crescimento dos seres humanos.

Fritz Perls era casado com Laura Perls, uma psicóloga que desenvolvia estudos na área de gestalt, um
ramo da psicologia da percepção. Foi a partir daí que ele adaptou as
teorias da psicologia gestáltica ao campo terapêutico. Para ilustrar
como seria uma gestalt do ponto de vista da percepção psicológica, basta
compreender este exemplo: uma pessoa está posicionada numa praia de
frente para o mar. A gestalt desse momento é a sua relação dinâmica com
a paisagem. Logo em seguida, passa uma gaivota que se torna o alvo da
observação dessa pessoa. Formou"-se, portanto, nova gestalt: a gaivota
torna"-se \emph{figura} desta gestalt, ou seja, o foco da atenção, seu
elemento principal em sentido e importância. O mar e o céu, assim como
ruídos e outros elementos restam como \emph{fundo}, representando tudo o
que está presente, mas ``por trás'', ou tendo importância secundária.

Toda gestalt sempre apresenta uma \emph{figura} e um \emph{fundo},
correspondendo, respectivamente, ao que é prioritário e secundário numa
dada situação. No exemplo mencionado, utilizamos a expressão ``abrir
gestalt'' ao momento em que a gaivota tornou"-se alvo da contemplação,
seja por prazer estético, curiosidade, ou qualquer outro motivo.
Definimos como ``fechar gestalt'' o momento em que a figura desaparece
ou nos desinteressamos dela. Tudo o que é vivo está permanentemente
``abrindo'' e ``fechando'' gestalts, num processo dinâmico e contínuo.

Fazendo uma comparação entre o estudo das gestalts e observações
clínicas de seus pacientes em terapia, Perls notou existir uma relação
muito próxima entre elas. Ele observou que, nas várias situações de vida
presentes no cotidiano de uma pessoa, surgem necessidades que seu
organismo determina como sendo mais ou menos emergentes. Este mecanismo
já descrito aqui, chamado de autorregulação espontânea, atua no sentido
de promover uma espécie de agenciamento organísmico, termo cunhado por
Perls para determinar uma função própria do nosso organismo. Assim, a
cada momento nosso organismo indica sua figura e seu fundo,
representando a necessidade física ou emocional que surge.

Por organismo, entendemos aqui que seja algo mais amplo que o nosso
corpo: nossa trajetória de vida, nossas demandas e desejos,
situadas em determinado tempo histórico e cultural. Esta teoria
organísmica é contrária às teses associacionistas, que buscavam
causa"-efeito. Ela constitui"-se basicamente pela busca entre as
inter"-relações existenciais entre os fenômenos, além de analisar as
funções psicológicas, mas tendo sempre o \emph{organismo} como um todo.

Assim, nosso ato de viver é um eterno processo de situações que somos
levados a lidar; que promovem a abertura e o necessário encaminhamento
de gestalts, pois, a cada momento, sempre existirá algo mais ou menos
prioritário em nossa vida. Se tivermos, por exemplo, vontade de comer, a
fome torna"-se a figura de nossa gestalt, nosso foco de atenção: está
aberta a gestalt. Precisamos saciar a fome, fechando a gestalt para,
imediatamente, surgir outra. Uma pessoa descobre que está atrasada numa
conta para pagar. Essa conta torna"-se figura da sua gestalt naquele
momento; ela a paga e a gestalt se fecha.

Outro exemplo seria alguém que trabalha num emprego que não lhe dá
prazer; essa gestalt só será fechada quando encontrar outra alternativa
de sobrevivência econômica mais prazerosa. Ou ainda, se ela ama alguém,
necessita de alguma forma aproximar"-se dessa pessoa e expor isso. Neste
caso, fechar a gestalt não significa necessariamente efetivar esta
relação, até porque depende da outra pessoa, mas como se manter atrelado
ou não àquele acontecimento. Ou o contrário, se uma pessoa vive um
relacionamento sem mais amor ou com muitos conflitos. Ou seja, todas as
situações de vida --- desde fisiológicas e corriqueiras do dia a dia, até
as mais complexas da existência --- estão em processo de abertura e
fechamento gestalts durante a vida. A partir dessas situações, pode"-se
imaginar exemplos de gestalts que precisam ser abertas ou fechadas em
todas as áreas e que envolvem os campos da saúde física, emocional,
financeira, profissional, familiar etc.

Outra descoberta fundamental de Perls foi perceber a necessidade vital
de se fechar as situações inacabadas para o equilíbrio e economia
energéticos. Há um consumo excessivo de energia vital quando não são
atendidas as necessidades da autorregulação que determinam o surgimento
das gestalts. Essa deficiência energética, como já havia dito Wilhem
Reich, é a principal responsável pelo aparecimento dos sintomas
neuróticos, bem como pela sensação de impotência e incompetência diante
da vida.

Para Friedrich Perls, a neurose se caracteriza pela perda da capacidade
de uma pessoa em conseguir fechar suas gestalts, deixando situações de
vida em aberto e sem solução. Tais situações podem permanecer inacabadas
por um longo tempo, pela dificuldade da pessoa em encontrar saídas
adequadas a elas. Esta condição, chamada de \emph{impasse} pela Gestalt,
é extremamente angustiante, pois paralisa a ação e impede a tomada de
decisão. Com isso, comumente a pessoa afasta"-se do presente, vivendo em
função do passado inacabado ou do futuro programado e repleto de
expectativas. Sua reserva energética á baixa, pois desperdiça muita
energia para manter suas gestalts abertas. Isso lhe dificulta abrir ou
fechar novas gestalts que vão surgindo, determinando acúmulo e
desorganização em sua vida no plano físico e/ou emocional"-psicológico.

Este ciclo vicioso produz as condições para que qualquer um sinta"-se
incapaz e impotente para a condução de sua vida e realização de seu
prazer. A existência adquire um peso e uma insatisfação terrível, já que
áreas fundamentais como o amor, a profissão, a sexualidade, por exemplo,
podem não estar sendo encaminhadas da maneira como se deseja, produzindo
o acúmulo de necessidades e anseios que não são atendidos. Segundo a
Gestalt, a condição neurótica emerge quando alguém não consegue ser
responsável pelo seu ato de viver, por não conseguir dar respostas às
suas necessidades cotidianas, mesmo que seja aceitar que não há uma
resposta possível para determinada situação. A noção de responsabilidade
adotada pela Gestalt não tem necessariamente um valor moral, mas
refere"-se ao fato de como cada pessoa participa ativamente da construção
de seu projeto existencial em sua busca pela liberdade.

No decorrer do processo terapêutico da Soma, os clientes adquirem
compreensão e energia para fechar e abrir suas gestalts. Isso por que,
objetivamente, o que torna uma vida insatisfatória é o fato de existirem
muitas situações de vida sem solução ou soluções insatisfatórias, o que
acaba por produzir mais conflitos e gasto de energia. A contribuição da
Gestalt nos auxilia a encaminhar de forma prática a terapia, pois atua
sobre as situações concretas que incomodam a vida das pessoas.

Para isso, trabalhamos sempre na avaliação e estudo do presente. Não
buscamos, como a maioria das técnicas terapêuticas tradicionais, um
retorno ao passado como forma de produzir a terapia. Devemos compreender
a história de vida do indivíduo, acompanhar seu desenvolvimento até a
fase adulta, mas sem se deter neste período ou fazer dele o foco de
nosso trabalho. Ao contrário, o processo terapêutico se dá no
aqui"-e"-agora. É nele que encontramos as gestalts abertas e seu
consequente esvaziamento da energia vital, bem como as estratégias e
possibilidades para fechá"-las. Esta posição fenomenológica justifica o
aqui"-e"-agora da Gestalt"-terapia, onde o presente é ao mesmo tempo o
passado, o presente e o futuro. É neste momento que ocorrem as
experiências, o reconhecimento de si e os projetos de cada um. O passado
está presente na minha fala, na minha respiração, nos meus movimentos e
na minha expressão. Ele é reestruturado em cada momento existencial, não
numa esfera estática da existência.

Desta forma, não fazemos uma terapia clínica tradicional, nem muito
menos confessional. A terapia se dá através da discussão do que se
vivencia em cada sessão. E, sobretudo, buscando a percepção (\emph{como}
cada situação acontece) no lugar da interpretação (\emph{por que} cada
situação acontece). Isto mostra o caráter objetivo e prático da Soma. Ao
final do processo terapêutico é que reunimos todos os \emph{comos} para
se chegar aos \emph{porquês}.

\section{Antipsiquiatria e a gramática da loucura}

A compreensão dos fatores políticos e sociais que bloqueiam a
originalidade e a autoregulação levou a Soma a se aproximar de pesquisas
sobre os mecanismos de poder responsáveis pela neurotização e até mesmo
pelo enlouquecimento das pessoas. As contribuições de Wilhelm Reich como
já foram aqui mostradas, estabelecem esta importante relação para a Soma
e seu processo terapêutico. Restava compreender como se dava esse
processo nas relações afetivas mais próximas e os mecanismos utilizados.

Sujeitados aos diferentes campos das práticas de poder, algumas pessoas
tornam"-se impotentes na condução de sua própria existência, tornando"-se
neuróticas. Dependendo do grau e intensidade dos conflitos, perdem o
sentido da vida e da potência de existir. Quando mais afetadas, chegam a
enlouquecer e outras se matam. As que sobram mantêm o sistema do poder
funcionando, corrompidas por ele, mas fazendo o seu jogo.

Os estudos que alteraram o significado e a importância da comunicação
sobre o comportamento humano levaram à criação de um novo e
revolucionário campo de saber chamado \emph{Antipsiquiatria}. Este termo
foi cunhado pela primeira vez em 1967 por David Cooper num contexto
muito preciso, e serviu para designar um movimento político radical de
contestação da prática psiquiátrica, desenvolvido especialmente num
período entre 1955 e 1975 na maioria dos países em que se haviam
implantado a psiquiatria e a psicanálise.

Este estudo sobre a pragmática da comunicação humana foi primeiramente
desenvolvido pelo movimento antipsiquiátrico ainda na década de 1950,
inicialmente nos Estados Unidos. Logo tornou"-se um dado científico
original na compreensão das relações de poder contidas nas interações em
diversos níveis da convivência humana. A importância dada a partir de
então aos vínculos afetivos e, sobretudo, ao tipo de comunicação utilizada
nestas esferas da sociabilidade, foi fundamental na compreensão das sutis
camadas do exercício do poder. Apesar de terem sido fundamentadas há
mais de cinquenta anos, estas pesquisas foram praticamente abandonadas
pela Psicologia e pela Sociologia atualmente.

Foi precisamente em 1956 que surgiu o trabalho decisivo neste campo,
realizado na Universidade de Palo Alto, na Califórnia, pelo antropólogo
e comunicólogo Gregory Bateson (1904-1980) e sua equipe, publicado com o
título de \emph{Sobre uma Teoria da Esquizofrenia}. Rapidamente
espalhou"-se pela Grã"-Bretanha com Ronald Laing e David Cooper; na
Itália com Franco Basaglia; e ainda nos Estados Unidos com as
comunidades terapêuticas, os trabalhos de Thomas Szasz e os assistentes
de Bateson.

Há muito tempo a psiquiatria tenta apresentar uma suposta origem para a
loucura, caracterizada por uma doença da mente e influenciada por
fatores hereditários, bioquímicos ou endócrinos, mas sem conseguir
precisar exatamente quais. A tentativa da psiquiatria em precisar um
\emph{locus} para as chamadas doenças mentais, seguindo o modelo
anátomo"-patológico da medicina tradicional, tornou praticamente
impossível qualquer resultado satisfatório. Além disso, durante muito
tempo, acreditou"-se em tratamentos terapêuticos lamentáveis e inúteis
como eletrochoques, choques insulínicos e internamentos longos e
violentos.

A Antipsiquiatria trouxe um olhar mais social e político para o
surgimento das doenças mentais; criticou as práticas adotadas nos
tratamento e na própria construção histórica da noção de loucura. Chegou
ao Brasil tardiamente e esteve presente nas lutas do movimento
anti"-manicomial e da reforma psiquiátrica, responsáveis por importantes
avanços nos tratamentos adotados por aqui. Suas ações trazem propostas
coerentes e possíveis, mas que muitas vezes foram boicotadas pela
psiquiatria tradicional e sua vulgar relação com as indústrias
farmacêuticas. Os antipsiquiatras denunciaram que a existência do que se
convencionou chamar de ``loucura'' foi por muito tempo utilizado pelos
sistemas autoritários como forma de perseguir e marginalizar pessoas,
excluindo"-as do convívio social. No passado e no presente, o Estado e a
sociedade fizeram uso de hospitais psiquiátricos pavorosos, que
funcionaram como verdadeiros cárceres, sem nenhuma função terapêutica.

Foi no estudo da esquizofrenia, a mais conhecida forma de desorganização
mental e vulgarmente chamada de loucura, que a Antipsiquiatria centrou
seus estudos. E foi a partir dela que surgiram as mais importantes
descobertas, ligadas, principalmente, ao estudo da comunicação humana
nos campos das relações afetivas. A Soma não trabalha com pessoas em
estado avançado de desequilíbrio emocional, como psicóticos, por
exemplo. Mas apenas com aquelas que ainda não perderam o contato com a
realidade, as que vivem conflitos e dificuldades características da
neurose. Por essa razão, utilizamos apenas alguns elementos da
Antipsiquiatria, sobretudo aqueles de função profilática para a neurose.

Uma das principais descobertas da Antipsiquiatria foi perceber que os
desequilíbrios emocionais decorrem de distúrbios na comunicação humana.
Existe um defeito na forma de se comunicar e de se relacionar,
normalmente utilizado na socialização, que leva inicialmente à confusão
e, consequentemente, à desorganização do pensamento e à dificuldade de
entendimento da realidade. Isso muitas vezes acontece desde a infância,
quando a criança recebe um tipo de comunicação paradoxal, chamada
\emph{duplo"-vínculo}.

Como a própria expressão indica, nesse caso, são enviadas duas mensagens
simultâneas, e uma sempre contrária à outra. É uma forma de afirmar e
negar algo ao mesmo tempo. Um sim e um não juntos, transmitidos pelo
mesmo canal de comunicação (como a fala) ou por canais de comunicação
diferentes (como a fala e a expressão gestual, por exemplo). Esse tipo
de comunicação paradoxal causa, inicialmente, um estado de confusão e
perturbação no entendimento do que está sendo comunicado. Geralmente,
vem seguido de sentimento de culpa: se aquele mal"-estar tem ou não
relação com uma determinada atitude que uma das partes tomou. A partir
da culpa, vem a dependência, quando a pessoa tenta de diferentes
maneiras atender à expectativa da outra. Seu uso contínuo pode levar à
completa perturbação na relação com o outro e com o mundo.

O duplo"-vínculo ocorreria, por exemplo, quando uma pessoa nos dissesse
sim e não ao mesmo tempo, se demonstrasse estar alegre e triste de uma
só vez ou evidenciasse nos odiar e afirmasse verbalmente seu amor por
nós. Como afirmamos, a dupla vinculação pode ser feita através de dois
canais de comunicação, afirmando pela fala e negando pela expressão
facial (alguém afirmar que concorda sobre algo com a fala e demonstrar
indiferença com uma expressão facial, por exemplo). Também pode ocorrer
em apenas um canal de comunicação, no paradoxo entre um conteúdo verbal
de uma fala e sua inflexão (alguém falar gritando que não está gritando,
por exemplo). Seja como for, seu uso contínuo e sistemático produz uma
interferência nos vínculos estabelecidos, onde comunicador e comunicado
se veem imersos num confuso e perigoso campo de entendimento e
não"-entendimento. A partir daí tudo pode ser verdade e/ou mentira, pois
nada está claro e sua própria dissimulação é condição constituinte do
jogo relacional.

O duplo"-vínculo estabelece uma situação paradoxal, que ocorre quando uma
pessoa se vê diante de mensagens de aceitação e rejeição. Tais mensagens
são simultâneas e contraditórias, de modo que quem as recebe fica
confuso pelo paradoxo. Esse quadro é muito comum em ambientes familiares
e com intensos vínculos amorosos. Foi justamente neste ambiente que a
Antipsiquiatria fez sua mais extraordinária e dramática descoberta ao
constatar que o duplo"-vínculo tem tanto mais poder de incidir sobre a
vida emocional, quanto maior e mais forte for a relação afetiva entre os
membros envolvidos. Ele surge principalmente nas relações familiares e
amorosas, tornando"-se um poderoso mecanismo de disciplina e controle.
Nestes casos, o amor se transforma em um poderoso instrumento de
dominação e neurotização dos indivíduos na vida em sociedade.

Vejamos um exemplo: uma garota de vinte e pouco anos deseja sair de
casa, morar com amigas, separada dos pais. Ela quer apenas vivenciar sua
escolha e sua liberdade sem que isso implique no corte do relacionamento
com a família. Num diálogo com a mãe, ela comunica sua decisão, quando
então, recebe um duplo"-vínculo, que a deixa confusa e culpada por sua
opção. A mãe diz estar feliz por sua filha resolver separar"-se dela e
seguir sua vida, mas ao mesmo tempo, não consegue conter as lágrimas ao
afirmar isso. Na verdade, está dizendo em palavras ser natural e
saudável a independência da filha, mas afirma também pelas lágrimas que
isto a fará sofrer muito. Mesmo questionada sobre o choro e a aparente
mágoa e desapontamento, a mãe insiste em dizer que está bem e aceita o
que a filha lhe comunica.

Embora não tenha sido dito, foi comunicado que a escolha da filha
provoca o sofrimento da mãe. O que produz sentimento de culpa nela que,
por isso, pode abdicar de seu desejo de liberdade e autonomia. Nesse
caso, foram utilizados dois canais de comunicação (a fala e a expressão
facial) na mensagem duplo"-vinculadora. Mas a mãe insiste e utiliza outro
paradoxo, agora utilizando apenas o canal de comunicação verbal,
dizendo: ``Seu pai vai entender também, como eu, o fato de você não
querer mais morar conosco, mas fale com cuidado, você sabe, ele já teve
um infarto, e ele gosta demais de você. Isso pode ser um duro golpe
nele.''

De qualquer modo, uma criança que foi educada utilizando"-se o amor
(através do duplo"-vínculo e da chantagem) como instrumento de dominação
de seus desejos por mais liberdade e autonomia, acaba tornando"-se
apática, insegura e, no limite, louca. Dificilmente se tornará uma pessoa
autônoma, decidida e corajosa. Isso porque, quase sempre, foi
questionada ou desqualificada em suas ideias e intenções. É provável que
o grau de apatia e dependências dos membros de uma determinada família
tenha tanto mais intensidade quanto maior o uso e frequência dos duplos"-vínculos.

No exemplo da garota houve mensagens paradoxais e duplo"-vinculadoras.
Não houve linguagem direta, sincera e objetiva, afirmando um sim ou um
não definitivo, ou mesmo explicitando o desejo contrário da mãe. Caso a
mãe mostrasse sua opinião, mesmo em desacordo com a escolha da filha,
mas de forma clara e sincera, não seria grave, pois haveria um impasse
explícito. O que prejudica a comunicação é o fato de existir tanto uma
como outra possibilidade, simultaneamente e em desacordo entre elas, e
ao mesmo tempo veladas. Comunicações semelhantes são utilizadas
constantemente no desenvolvimento dos indivíduos nas sociedades
hierarquizadas e, com isso, a confusão vai deformando, alterando a
compreensão dos fatos e modificando o seu comportamento. Esta é a forma
de comunicação utilizada pelas famílias na grande maioria das pessoas
que se tornam neuróticas.

Devido à ausência de sinceridade, as relações tornam"-se poluídas de
mentiras, chantagens e duplos"-vínculos. No Brasil, existe uma expressão
largamente utilizada, muitas vezes em sentido duplo"-vinculador: é o
famoso ``tudo bem''. Em muitas situações as pessoas não gostam ou não
concordam com determinadas coisas, mas mesmo assim, dizem: ``tudo
bem\ldots{}''. Às vezes sente"-se que não está tão bem assim. Sentimo"-nos,
então, confusos ou culpados por nossos atos e escolhas na relação com
àquela determinada situação. Outro dispositivo frequente nas interações
sociais é a ironia, quando são utilizadas palavras que manifestam o
sentido oposto do seu significado literal. Diante da dificuldade em ser
franco com alguém, a ironia serve para afirmar o contrário daquilo que
se quer dizer ou do que se pensa. Seja como for, quando estas
comunicações são estabelecidas entre pessoas que não possuem um forte
componente afetivo, seu efeito é bastante menor. O duplo"-vínculo só
causa confusão, culpa e dependência quando existe a ameaça subjetiva de
perda afetiva.

Outro aspecto importante do duplo"-vínculo é o fato dele ser sempre
bilateral. Ele só funciona quando o comunicador e o comunicado
mutuamente entram num jogo. Portanto, a única forma de evitar o
duplo"-vínculo é uma das duas pessoas da relação não praticar o seu. Na
Soma, estudamos os duplos"-vínculos recebidos por uma pessoa que a
deixaram confusa e dependente em relação a outras. Mas estudamos,
sobretudo, os duplos"-vínculos que ela aplica sobre as pessoas que ama e
com as quais se relaciona, confundindo"-as e deixando"-as fracas, sob sua
dependência. Toda pessoa que foi duplo"-vinculada torna"-se grande
duplo"-vinculador. É assim que se opera a cadeia autoritária no plano
microssocial, na reprodução quase espontânea dos mecanismos de poder
presentes em relações próximas e envolvidas por importantes laços
afetivos.

São estes jogos sutis de poder, utilizados na comunicação diária, que
serão prioritariamente estudados na micropolítica do cotidiano. Nos
grupos de terapia, procuramos incentivar uma nova maneira de convivência
entre as pessoas, em que a comunicação seja baseada em uma política de
sinceridade. Logo, ao denunciar o duplo"-vínculo e seus efeitos, estamos
combatendo o mecanismo primário do autoritarismo nas relações
interpessoais. É assim que trabalhamos sobre os mecanismos
comunicacionais das relações humanas, onde surgem os primeiros indícios
da neurose.

O duplo"-vínculo é uma poderosa arma da dominação social porque acontece
no dia a dia, na relação amorosa, na relação pai e filho. Acreditamos
que é preciso transformar a forma convencional de organização social,
baseada nos padrões tradicionais de organização familiar, onde o amor é
utilizado para gerar culpa e ressentimento. São nelas, fundamentalmente,
que se operam os mecanismos de docilização dos indivíduos. Como afirmou
Wilhelm Reich, ``a família espelha e reproduz o Estado'', criando as
condições básicas de disciplina e controle, que mais tarde se ampliam na
vida social dos indivíduos.

O homem, sendo um animal gregário, necessita de outros para a sua
sobrevivência. Ele se constitui na interação social, cultural e afetiva
entre outros seres humanos. Portanto, alguma forma de agrupamento
humano, como a família, precisa existir, mas com uma ética de
convivência diferente da que conhecemos. Uma nova família, em que o amor
não seja utilizado para dominar e neurotizar, mas apenas para ser vivido
e desfrutado.

\section{Por uma psicologia da autonomia}

A Soma é uma terapia explicitamente libertária. Adotamos há muitos anos
a designação de \emph{uma terapia anarquista} como forma de deixar claro
uma perspectiva ético"-política na condução e objetivos terapêuticos. Os
conceitos e as vertentes teóricas até aqui apresentadas se articulam em
torno da compreensão de que o ato terapêutico deve levar em consideração
não apenas o indivíduo isolado, mas também suas relações sociais e as
práticas de poder nelas envolvidas. Nosso entendimento de saúde,
portanto, caminha junto com a construção de sociabilidades mais livres e
mais horizontais, refletindo um modo de vida libertário no aqui"-e"-agora.

Sabemos que a neurose, além de produto das relações autoritárias que
começam na família tradicional, é também a garantia para a manutenção da
sociedade hierarquizada. Os mesmos mecanismos de poder que nos afetam
e reproduzimos, nos afastam de nossa autorregulação e de nossa
unicidade. Sem nos darmos conta, também sacrificam"-se projetos e sonhos
individuais para ter melhor aceitação das pessoas e manter a máquina
social funcionando. A neurose acaba por condenar a vida a ser apenas uma
peça a mais de uma engrenagem instituída e oficializada: o sistema
capitalista.

Fica claro, assim, que somente uma mudança na ética de convivência entre
os indivíduos pode quebrar o círculo vicioso das relações baseadas no
jogo dominador"-dominado. Somos neurotizados para sermos mais facilmente
dominados e, o que é mais grave, nos tornamos também agentes
neurotizantes das pessoas com quem nos relacionamos. Para nós na Soma,
não temos dúvida de que só é possível modificar esse esquema de
reprodução e perpetuação da disciplina e do controle através de uma vida
libertária.

A queda do bloco comunista do Leste Europeu nos fins dos anos de 1980
desencadeou a ruptura do sonho marxista de socialismo de Estado como
contraponto ao capitalismo. Restou ao anarquismo, ou socialismo
libertário, ser a única ideologia que se apresenta contrária ao
capitalismo no presente, caracterizado em sua dimensão globalizada e
hegemônica no mundo, capilarizada em diferentes esferas da sociedade.
Através da vida libertária é possível pensar uma atitude rebelde e ativa
no cotidiano, e buscar novas formas de atuação no campo ético e
político.

O anarquismo vem sendo combatido desde o final do século \versal{XIX}, tanto
pelo capitalismo como pelo socialismo autoritário marxista. A principal
forma de desqualificação foi associá"-lo ao sentido de bagunça, desordem,
confusão. Esse sentido pejorativo foi utilizado pelos que acreditavam na
necessidade de um poder instituído, uma autoridade para governar os
homens. O governo seria a garantia de uma sociedade organizada e justa.
Mas a história prova o contrário, os governos serviram para manter o
privilégio das minorias, produzir guerras e manter o jogo autoritário
sobre as pessoas.

A palavra anarquia vem do grego e significa ``ausência de autoridade ou
governo''. Para que o anarquismo aconteça é necessário, ao contrário do
que se propaga, que a vida libertária ocorra através da auto"-organização
individual e social. Porém, esta organização jamais deve ser exigida ou
mesmo determinada por algum agrupamento de pessoas, sejam partidos,
governos ou qualquer forma de organização autoritária. Mas, a partir de
organizações livres, auto"-organizadas e federadas, criadas por pessoas
que construam sua existência de maneira autônoma. A anarquia só poderá
emergir a partir de uma pluralidade de auto"-governos inscritos~em
comportamentos individuais e coletivos.

O anarquismo não tem uma proposta fechada de organização social. Não é
um objetivo a ser alcançado como um paraíso libertário, mas, sim, um
processo de transformação a ser construído cotidianamente. A prática
anarquista em direção à construção de sociabilidades mais livres se dá no
presente, na eliminação do poder autoritário entre as pessoas, no
respeito às singularidades e individualidades e na prática social
autogestiva.

Nas práticas libertárias, não estão em jogo a tomada e posse do poder,
mas, sim, a sua destruição, enquanto poder centralizado e hierarquizado,
ao mesmo tempo em que haja uma socialização dos poderes. Qualquer
mudança na sociedade só é possível se for feita de baixo para cima,
nunca ao contrário, imposta por um governo, mesmo que este se
autodenomine ``socialista''. A vivência libertária acontece na
construção de uma nova ética nas relações pessoais e sociais, em que a
dominação seja substituída pela solidariedade. Isto não precisa ser
feito no futuro, a partir do equacionamento das disputas de poder ou do
fim do autoritarismo, o que seria uma abstração; mas, sim, a partir de
agora, no dia a dia das pessoas.

Desta forma, a Soma utiliza no decorrer de sua prática terapêutica as
referências libertárias do anarquismo, para mostrar outra possibilidade
para as pessoas de relacionarem"-se com o mundo de maneira distinta da
qual normalmente nos relacionamos. Esse jeito novo é a aposta numa
postura de vida autêntica, potente e prazerosa no cotidiano. Cada pessoa
pode e deve exercer aqui"-e"-agora sua diferença para tornar"-se um ser
autônomo, autorregulado e responsável por sua vida e seus atos.

Nós, somaterapeutas, procuramos transformar nosso dia a dia na prática
libertária. Em nosso trabalho, o anarquismo está presente como
referencial metodológico para mais fácil identificar os jogos de
autoritarismo, fonte da neurose. Entretanto, não procuramos fazê"-lo de
forma proselitista, até porque a descoberta em cada um de sua ideologia
é uma condição pessoal. Mas mostramos como uma pessoa pode lutar por sua
liberdade mesmo dentro de uma sociedade autoritária, através da prática
libertária.

O anarquismo liga"-se diretamente a todas as teorias psicológicas em que
nos baseamos. Assim, uma pessoa que vive de forma anarquista o seu
cotidiano luta para não se submeter a qualquer forma de autoritarismo e,
sobretudo, não exercer mecanismos de poder em suas relações amorosas,
através de chantagens ou duplos"-vínculos. A fala franca se apresenta
como ato de coragem para o exercício da comunicação sincera, clara e
objetiva. Sua couraça deverá estar mais flexível, menos rígida e
cronificada, uma vez que busca enfrentar e lutar, tanto pessoal como
socialmente. Isto lhe reserva energia para fechar suas gestalts e poder criar
e amar de forma e jeito próprios.

A prática anarquista e sua crítica permanente ao poder centralizado
servem de referencial à nossa metodologia, tanto na forma em que o grupo
trabalha (autogestão), quanto na relação terapeuta"-grupo. O
somaterapeuta participa diretamente da dinâmica do grupo, se posiciona e
recebe dos outros membros críticas e sugestões sobre seu trabalho. Ele
participa do processo coordenando o grupo, como um membro do processo
autogestivo, que exerce sua liderança de forma não"-autoritária.

Desta forma, os demais integrantes podem concordar ou não com as suas
ideias. No caso de outras terapias, é possível que o cliente passe anos
sem saber nada da vida do terapeuta e este pode ter uma ideologia
completamente contrária à sua, com objetivos de vida opostos. O
somaterapeuta é um líder circunstancial atuando da maneira mais
horizontal possível. Ele compartilha os seus conhecimentos com o grupo
para que os participantes possam, num período predeterminado (um ano a
um ano e meio), ser capazes de adquirir autonomia emocional sobre suas
vidas, sem desenvolver nenhuma dependência em relação à terapia, ao
terapeuta ou ao grupo.

Na Soma, defendemos a liberdade como posicionamento político e o prazer
como referencial ético, constituindo assim, um modo de viver libertário.
Ao adotarmos uma posição ideológica anarquista, estamos nos colocando no
mundo de forma a não aceitar nenhuma forma de autoritarismo de pessoas
(pelo afeto, nos relacionamentos amorosos e na família) e de Estado
(pela violência, por imposições econômicas e morais).

Na dinâmica autogestionária da Soma, os membros do grupo experimentam um
jeito novo e instigante de conviver e produzir uns com os outros.
Exploramos os potenciais criativos, afetivos e produtivos, sempre
baseados na sinceridade, na solidariedade e na cumplicidade. Esta
transformação, experimentada no microlaboratório social dos grupos de
Soma, é estendida às relações cotidianas de seus participantes. É assim
que se processa a finalidade terapêutica e social da Soma. Isto não
existe em qualquer outra terapia. E é exatamente daí que vem o espírito
libertário e lúdico do processo: a oportunidade de se experimentar como
se é, para viver o seu prazer, a sua diferença, e sem que isso recaia
sobre o outro.

Mais do que uma simples terapia, a Soma é pedagogia, auxiliando as
pessoas a conhecer seu soma, suas características e sua originalidade
única. Em síntese, a Soma se propõe a ser uma pedagogia para a
liberdade. Esta proposta clara e aberta de trabalho faz com que os
participantes que mais aproveitam o grupo sejam os que procuram a Soma
para mudar a vida e não os que vêm atrás de alívio para suas neuroses.

Esta transformação na forma de conduzir a existência busca criar uma
vida sem submissão e sem autoritarismo. Assim, a Soma não é uma terapia
para simplesmente adaptar as pessoas à sociedade tal qual ela é e
funciona. O nosso conceito de saúde aponta para o desenvolvimento da
criatividade que leve à construção de novas sociabilidades mais livres e
justas, em que o ato de viver não se limite apenas a sobreviver.

Na criação da Soma, a autogestão como princípio libertário foi
incorporada ao processo terapêutico por Roberto Freire como prerrogativa
do funcionamento do coletivo. A dinâmica de grupo vivificada em
autogestão estimula que as decisões, encaminhamentos e escolhas de cada
membro sejam valorizadas e realizadas sob outro paradigma de associação,
e com isso levar os participantes a elaborarem novas vias associativas e
sociabilidades libertárias menos hierarquizadas e consequentemente menos
autoritárias.

Viver o processo de um grupo de Somaterapia é dispor"-se a uma
(re)invenção de si mesmo. O grupo representa um espaço de elaboração de
si, no qual cada membro busca entender seu funcionamento
emocional"-psicológico, simultaneamente ao seu funcionamento ético,
social e político. Mas ao perceber estas formas de atuar no mundo, este
mesmo grupo também possibilita a mudança de postura e a criação de novos
modos de existir. Seriam, portanto, uma função diagnóstica agindo em
simultaneidade com uma função transformadora: na medida em que se
percebe uma questão, são formuladas estratégias de mudança; isso amplia
a percepção, criando novos diagnósticos e novas possibilidades de
mudanças.

Desta forma, os espaços coletivos dos grupos de Soma atuam no sentido de
promover novas formulações sociais, menos hierarquizadas e mais
libertárias. Este projeto segue próximo ao que Reich defendia como o
lugar possível de uma psicologia transformadora, ou seja, na confecção
de sociabilidades que estabeleçam um contraponto ao capitalismo. Caso
contrário, os mecanismos que produzem a neurose seguem seus cursos, e a
clínica corre o risco de transformar"-se em uma mera mantenedora destes
mecanismos. Neste caso, a psicologia atua como paliativa, tornando sua
prática um espaço circular e sobrecodificante.

\chapter*{A metodologia da Soma}
\addcontentsline{toc}{chapter}{III. A metodologia da Soma}

\section{A formação de um grupo de Soma}

Os conceitos até agora apresentados mostram porque a Soma é um processo
terapêutico"-pedagógico realizado em grupos, com uma ênfase no trabalho
corporal, de duração predeterminada e de conteúdo político explícito: o
anarquismo. Trataremos a seguir de delinear seu funcionamento
metodológico e seu processo terapêutico.

Os grupos de terapia formam"-se geralmente após uma Oficina de Soma.
Estas oficinas normalmente ocorrem em um fim de semana e são compostas
por demonstrações práticas dos exercícios, além de aulas teóricas sobre
as bases científicas da Soma. A Oficina busca fornecer um panorama
sintético de nosso trabalho, informando às pessoas interessadas o
material necessário para optar se desejam participar de um grupo. As
pessoas que procuram as oficinas são geralmente jovens de diversas
idades e profissões, mas com uma identificação comum aos temas da
liberdade e da autonomia. Não é necessário nenhum conhecimento prévio
para participar delas.

Ao terminar a oficina, seus participantes são convidados a iniciar um
novo grupo, que terá a duração em torno de um ano a um ano e meio. Este
grupo permanecerá aberto pelos três primeiros meses, podendo também
integrar"-se a ele pessoas que não fizeram a oficina. Após este período
inicial da formação do grupo, não entra mais pessoas e ele é fechado
para que se possa ser desenvolvida a sua dinâmica e para seguir seu
processo.

Geralmente os grupos têm em torno de 12 a 16 pessoas, podendo variar
para mais ou para menos. Começamos os grupos com um número maior de
participantes, pois há uma queda do número de pessoas ao longo do
processo. É importante que o grupo não chegue ao final muito pequeno, já
que afetaria a execução dos exercícios e a dinâmica de grupo. O tempo de
duração do processo também é afetado pelo número de pessoas que
efetivamente chega ao final. Grupo maiores tentem a dura um pouco mais,
mas nunca ultrapassando dezoito meses.

\section{As sessões de terapia}

O grupo trabalha em quatro sessões de terapia, distribuídas ao longo de
cada mês em datas e horários pactuados entre todos os integrantes. Cada
uma delas tem por volta de três horas de duração, totalizando doze horas
de terapia mensais. Na primeira metade de uma sessão de Soma, são
realizados \emph{exercícios corporais}.

A Soma tem mais de quarenta exercícios corporais próprios. São
diferentes práticas corporais/vivenciais, quase todas com função
bioenergética, voltadas à mobilização da energia vital, seguindo as
indicações de Wilhelm Reich. Além disso, cada uma delas busca atuar
sobre um determinado ``campo de vida''. Assim, elas agem sobre o
comportamento de cada um, despertando elementos sobre sua comunicação,
agressividade, tomada de decisão, confiança, medos e uma série de outras
emoções e sentimentos. Os exercícios da Soma também fazem emergir
questões relativas às práticas sociais, políticas e éticas entres os
membros do grupo.

Dessa forma, os exercícios oferecem elementos a cada membro do grupo e
ao conjunto dos participantes. A variedade desses elementos e suas
implicações irá depender da proposta do exercício em questão, de como
cada um é afetado e reage a ele, ao período do processo terapêutico e do
grupo. Se em algumas das vivências a abordagem se volta para uma
percepção mais individual, encontramos, em outras ocasiões, o grupo como
um todo sendo objeto de investigação. Os modos de funcionamento da
dinâmica do grupo tendem a refletir questões relativas aos atores ali
presentes. Assim, se o grupo encontra"-se coeso e envolvido no processo,
isto se deve aos mecanismos encontrados por todos na formulação de
impasses, conflitos e práticas de poder. Por outro lado, uma dinâmica de
grupo na qual haja evasão, dispersão ou recorrentes conflitos revela a
dificuldade daqueles atores em elaborar soluções para as questões
apresentadas. Apesar de não ser pré"-definida a distribuição dos
exercícios ao longo do grupo, existem aqueles que são realizados
preferencialmente no começo, no meio ou no final do processo.

Os exercícios têm origem no teatro, em danças e jogos infantis, mas são
utilizados com objetivos específicos à terapia. Porém, existe uma raiz
comum a todos: a ludoterapia. Essa terapia, aplicada a crianças,
baseia"-se em levá"-las a brincar e jogar com brinquedos e jogos
padronizados, de modo que se possa revelar, na maneira e forma que se
relacionam com eles, os elementos conflitantes em suas vidas, que depois
são avaliados com seus pais ou responsáveis.

Os exercícios também são jogos e brincadeiras, onde em sua maioria não
se usa a comunicação verbal. Neles, os adultos são levados, através da
ludicidade, a comunicar corporalmente suas dificuldades e bloqueios.
Costumamos dizer que uma pessoa, quando faz isto, dá ``bandeira'' de
seus problemas. Tudo o que ela procurou esconder, controlando seus
movimentos, expressões e comportamentos, revela"-se através de alterações
do sistema neurovegetativo (tonturas, sudoreses, desequilíbrios,
náuseas) ou através das sensações e percepções que ocorrem durante os
exercícios. Essas ``bandeiras'' serão estudadas na fase seguinte da
sessão.

De forma geral, os exercícios funcionando como disparadores de sensações
e têm por finalidade servir como ponto de partida para problematizações
que serão realizadas e aprofundadas no segundo momento de uma sessão de
Soma: a \emph{leitura do exercício,} quando trabalhamos a comunicação
verbal. Assim, os participantes do grupo dispõem"-se em círculo, sentados
no chão, e vão relatando, com a ajuda da memória somática (tudo o que o
corpo todo registrou, e não apenas o pensamento), as sensações e
percepções captadas durante a fase anterior. Este é o momento de
decodificar em palavras o que foi vivenciado corporalmente, quando cada
membro do grupo busca perceber as questões evidenciadas nos exercícios
vivenciais. Cada pessoa vai falando o que sentiu, desde dificuldades
(medos, lembranças, bloqueios etc.) até descobertas de prazeres,
encantamentos e situações novas de vida. Pode"-se falar também das
``bandeiras'' observadas nos outros, quando ocorrer algo que possa
interessar a estes. Porém, quando isso acontece, não se procura
julgá"-los ou qualificá"-los, mas se faz com o objetivo de facilitar a
compreensão sobre seus bloqueios. Assim, a leitura não é a busca da
verdade, mas o exercício da sinceridade, a forma que cada um viu e
sentiu, a si e aos outros por meio de feedbacks.

Outra característica da leitura é trabalhar de maneira gestáltica, ou
seja, sempre buscando \emph{como} as bandeiras aconteceram em detrimento
do \emph{porquê} elas ocorrem. Em vez de interpretá"-las, os
participantes são estimulados a compreender o funcionamento dos
mecanismos neuróticos denunciados no exercício. Procura"-se descrever as
sensações e percepções do presente e não fazer interpretações ligadas ao
passado. Essas descrições vão ajudar, ao final da terapia, a fechar
interpretações definitivas, levando em conta todos os ``comos'' para se
chegar aos ``porquês''.

O terapeuta, que esteve observando e circunstancialmente intervindo na
leitura, posicionado no círculo junto ao grupo, porém sem ter
participado diretamente do exercício, finaliza o trabalho fazendo o
\emph{fechamento da sessão}. Ele deve fazer uma síntese do material
trazido pelo grupo em função do exercício. Analisa suas causas
psicológicas e sócio"-políticas, e prenuncia como poderão ser enfrentadas
pela ideologia e pela prática da Soma. O grupo recebe, então, material
organizado pelo terapeuta que deverá auxiliá"-lo no entendimento e na
compreensão dos mecanismos atuais da sua neurose. Encerra"-se aí uma
sessão de terapia, e como foi mostrado, apresenta estas três etapas:
exercício, leitura e fechamento.

Nas quatro sessões de terapia mensais, os temas nelas trabalhados buscam
uma correlação entre si. Ao longo do processo, estas várias sessões irão
fornecendo peças de um ``quebra"-cabeça'' que cada um vai montando sobre
suas características e conflitos. Estas informações auxiliam cada
participante a ter uma melhor clareza de seu funcionamento
emocional"-psicológico, além de atuarem terapeuticamente.

Há alguns temas que estão mais presentes que outros, surgindo em
diferentes sessões. É o caso, por exemplo, do tema que estabelece uma
relação entre o uso da agressividade e a violência. A Soma, por ser um
processo que visa à construção de práticas livres no cotidiano, investe
fortemente na noção de luta e enfrentamento como prerrogativa para
incrementar uma postura afirmativa na vida. Acreditamos que é necessário
saber dosar nossa agressividade nas inúmeras circunstâncias de vida para
que ela, acumulada, não se transforme em violência. Sempre pensada como
uma agonística, a noção de luta para a Soma não significa competição,
quem ganha ou quem perde, mas um contínuo e permanente embate
diante da vida. Análise semelhante ocorre também, por exemplo, com os
temas da espontaneidade, da criatividade e da comunicação que figuram em
distintos momentos do processo do grupo.

\section{A dinâmica de grupo autogestiva e os grupões}

Além das sessões, o grupo encontra"-se no mínimo uma vez por mês, numa
reunião sem a presença do terapeuta chamada \emph{grupão}. Nos grupões,
os participantes discutem a própria terapia, o trabalho do grupo e do
terapeuta, além de aprofundar estudos científicos correlatos à terapia.
Esses encontros são fundamentais para a prática da Soma, sobretudo
porque neles se experimenta trabalhar em autogestão: aprendizado de uma
nova forma de produção social, baseado principalmente na sinceridade e
na solidariedade anarquistas.

A autogestão é um meio de produção tipicamente libertário. Configura"-se
como uma prática que se dá na relação permanente de fragmentação do
poder. Pressupõe a auto"-organização; o apoio mútuo; a valorização das
diferenças individuais e, ao mesmo tempo, a tentativa por formar
unidades entre diferenças; a recusa pelas práticas hetorogestoras,
hierarquizadas, centralizadoras e autoritárias da organização social.

O termo \emph{autogestão} é relativamente novo, tendo sido incluído nos
dicionários franceses na década de 1960, pós"-Maio de 1968. No entanto,
sua concepção de organização é antiga e está intimamente ligada às
tradições libertárias. Apesar de a palavra autogestão, hoje em dia, ser
utilizada de diferentes modos e em distintos contextos, seus princípios
e suas práticas estão relacionadas à história dos anarquismos.

Ela está implicada com a ideia de acontecimentos concretos e criação de
sociabilidades de experimentação social, que se instituem e se constroem
por si mesmas. A autogestão é, portanto, um processo e uma perspectiva
de transformação individual e social. É um movimento, produto da
experiência de vitórias e de derrotas; é um amplo procedimento de
experimentções em todo o conjunto da vida social que visa à construção
de relações mais livres.

Na Soma, adotamos a autogestão como princípio básico de produção da
terapia. Todos participam, contribuindo e recebendo apoio uns dos
outros. O terapeuta serve, neste caso, apenas como um catalisador,
fomentando e apoiando a construção da dinâmica de grupo. Na prática
autogestiva, o grupo é levado a experimentar um relacionamento social
novo, no qual haja uma crítica permanente ao autoritarismo e ao
centralismo. Todos são responsáveis pela organização prática do
trabalho. No grupão, também são aprofundadas as relações entre os
membros participantes, solucionados impasses e conflitos, sempre através
do uso da sinceridade.

As características dos grupões são vastas e variadas, dependendo de como
cada grupo apresenta sua própria dinâmica, porém estes encontros
geralmente possuem forte poder terapêutico. Na construção da dinâmica de
grupo, vivificada nos grupões, costumam surgir saldos de nossa formação
autoritária como a submissão, o oportunismo, a apatia, a dominação etc.
Eles são avaliados criticamente na dinâmica e o grupo toma decisões por
consenso. Como trabalhamos sob uma perspectiva libertária, somos contra
o autoritarismo de a maioria ter mais poder sobre a minoria, por isso
buscamos a decisão sem voto. Através da ação libertária, as decisões do
grupo são tomadas sempre levando em conta os direitos e objetivos de
todos, aceitando e respeitando as diferenças para se chegar à unidade na
diversidade. O impasse, nas decisões, só ocorrerá se houver no grupo
alguma vontade autoritária. Assim, na dinâmica de grupo numa perspectiva
autogestionária, como o próprio nome indica, o trabalho terapêutico é
desenvolvido e liderado por todos que pertencem ao grupo.

A figura do somaterapeuta como líder circunstancial deixa de existir
quando a terapia acaba. Ele é uma pessoa que detém conhecimento e
vivência suficientes e necessários, capazes de facilitar o processo
terapêutico de outras. O somaterapeuta, como qualquer outra pessoa,
também é suscetível à neurose, uma vez que vive na mesma sociedade
neurotizante. A Soma elimina a imagem mitificada que geralmente se faz
em torno do terapeuta. A maioria das práticas de terapias existentes
sustenta esta imagem e, por esta razão, espera"-se que o terapeuta tenha
o poder de curar neuroses já que é uma pessoa livre delas. Não
acreditamos nem aceitamos essa visão autoritária e irreal.

O somaterapeuta também deve se colocar, deve ser foco e alvo de críticas
como qualquer outro participante do grupo. Seu papel ali é de catalisar
o processo terapêutico da maneira mais horizontal possível, assim como
estimular e apresentar instrumentos para que os membros do grupos atuem
de maneira ativa na produção coletiva da terapia. Mesmo que não esteja
diretamente envolvido nas problematizações em torno de sua própria
terapia, não deixa de expor questões de vida, posicionar"-se diante dos
acontecimentos sociais e políticos do momento e romper, com isso,
qualquer noção de neutralidade.

Além dos grupões, existe ainda o que chamamos de \emph{marrom}. São
encontros sociais variados que o grupo, ou parte dele, realiza. É, por
exemplo, irem juntos ao cinema, festas, praias ou qualquer outra forma
de convívio social. Estes encontros são muito importantes para a
socialização e integração afetiva dos participantes. Esta é uma prática
condenada por outras terapias, como a Psicanálise, por exemplo, que a
chama de ``acting out''.

As sessões de Soma são realizadas geralmente em salões. São espaços de
dança, de ginástica ou de esportes onde o grupo se reúne para realizar as
sessões de terapia. Para se aproximar mais dos objetivos libertários que
a Soma defende, realizamos vivências fora das cidades, chamadas de
\emph{Maratonas de Campo.} Durante um fim de semana, o grupo convive,
produz e faz terapia em contato direto com a natureza. Os grupos que
funcionam no sudeste do país fazem suas maratonas em Visconde de Mauá,
no Rio de Janeiro. Para os grupos do nordeste trabalhamos na Chapada
Diamantina, na Bahia. E os grupos da região sul, nos cânions de
Itaimbezinho e Fortaleza, no Rio Grande do Sul.

Essas regiões têm características comuns que aproveitamos como
componentes dos exercícios específicos da maratona. Contato com rios,
cachoeiras, trabalhos com argila, passeios pela mata etc. são utilizados
de maneira terapêutica. Além de promover uma imersão do grupo, que
inclui a viagem, a hospedagem e a convivência por três dias, estas viagens
possibilitam fazer emergir diferentes questões terapêuticas em função do
contato direto com a natureza. O contato com a água gelada, a diferença
de relevos e altitudes, o prazer estético são alguns dos elementos que
trazem ganhos significativos ao processo de cada um e à dinâmica do
grupo. Se a vida nas cidades traz um certo controle das formas
que nos deslocamos e funcionamos, a natureza oferece um conjunto de
estímulos e situações que escapam do que está posto, requisitando uma
presentificação, uma atenção voltada ao aqui"-e"-agora. Caminhar sobre as
pedras de um rio, por exemplo, requer observar e decidir a cada instante
as direções do próprio deslocamento.

São realizadas até três maratonas de campo. Eles ocorrem durante o
período dos exercícios, a primeira em torno do sexta mês e a segunda
quando acaba a bateria de exercícios. Por fim, eventualmente realizamos
mais uma maratona de campo quando acaba a terapia. Elas servem ainda
como experiência autogestionária do grupo, sendo avaliados os processos
de organização coletiva para a viajem, a hospedagem e a convivência
social durante o período.

\section{As cadeiras quentes e o encerramento do processo}

Trabalhamos com uma média de quarenta e cinco exercícios terapêuticos
durante o processo do grupo. Em média, no sexto mês realizamos a
primeira maratona de campo e no décimo mês a segunda. Quando acaba esta
fase dos exercícios na Soma, inicia"-se a fase final da terapia chamada
\emph{Cadeira Quente}. Este é o momento de organizar o material
terapêutico dos membros do grupo de Soma. A esta altura, cada pessoa já
atingiu o claro entendimento da prática da dinâmica de grupo numa
perspectiva autogestiva. Também já criaram"-se laços e vínculos de
cumplicidade, respeito e confiança entre os membros do grupo, assim como
a percepção de como cada um opera neste micro"-laboratório social.

Na cadeira quente, cada sessão de três horas é dedicada a apenas um
participante do grupo, com o intuito de reunir os principais elementos
que representaram a passagem desta pessoa neste grupo. A cadeira é uma
criação original da Soma. Ela surgiu bem depois que Freire começou a
nomear seu trabalho de Soma, quando percebeu ser incompleto o término da
terapia após o final dos exercícios. Então, imaginou algo semelhante ao
jeito como trabalhava Friedrich Perls no estudo dos sonhos de seus
clientes na Gestalt"-terapia, através da chamada \emph{hot seat}. Apoiando
apenas na disposição cênica do trabalho gestáltico, na qual o
``cadeirado'' e o terapeuta sentam"-se lado ao lado e o restante do grupo
posiciona"-se em forma de plateia diante dos dois, a cadeira quente da
Soma não trabalha na análise dos sonhos, mas nos percursos de cada um ao
longo da terapia. Divide"-se em quatro etapas: \emph{indução},
\emph{afirmações}, \emph{perguntas} e \emph{fechamento}.

Durante as três horas dedicadas àquela pessoa, são apresentadas pelo
terapeuta e pelos membros do grupo todas as impressões colhidas a seu
respeito em mais de um ano de convivência. É um momento privilegiado na
autogestão da Soma, em que o terapeuta e os membros do grupo trabalham
juntos, no sentido de proporcionar, a cada um, o entendimento dos fatores
psicológicos, sociais e políticos que inibem a expressão de sua
originalidade única. Nas quatro etapas que compõem a cadeira quente, o
terapeuta e os demais membros trabalham a fim de proporcionar o maior
número possível de informações ao ``cadeirado''.

Não se trata de um julgamento, muito menos de um conjunto de conselhos
que o outro irá seguir, mas sim da reunião dos elementos e
características desta pessoa no grupo. É a busca por criar juntos uma
cartografia de si, na qual procura"-se identificar as particularidades da
couraça emocional de cada pessoa, assim como entender como ela lida com
suas gestalts, estabelece seus paradoxos comunicacionais, opera nas
sociabilidades do ponto de vista ético, estético e político. Enfim,
procura"-se, nas cadeiras quentes, traçar os principais elementos que nos
constituem e nossos processos de subjetivação.

Isto se faz através de uma política da sinceridade. É algo semelhante ao
que Michel Foucault chama de \emph{parrésia} ou a coragem da verdade: um
ato de força, franqueza e ousadia para falar ao outro. A parresía é o
termo em grego para designar a coragem de se dizer o que se pensa, em
expor tudo e de falar francamente. Foucault, em sua obra \emph{O governo
de si e dos outros,} trata desta antiga noção e seu uso político desde
os tempos da politeia (constituição) e da democracia na Grécia. A
palavra certa, proferida no momento adequado, pode ser um poderoso
instrumento para romper pactos silenciosos, muitas vezes camuflados em
ironias e sarcasmos. Quando a verdade é proferida sem malícia, não se
guarda rancores. Se houver respeito e confiança para afirmar o que se
pensa, o exercício da sinceridade não ofende. Normalmente, no período da
cadeira quente, o grupo já adquiriu as condições para que todos possam
ser parrésicos e autênticos uns com os outros, sem que isso cause
qualquer constrangimento.

Assim, este trabalho das cadeiras quentes, realizado com bastante
cuidado e carinho, representa o período mais gratificante da Soma, assim
como o mais belo e encantador. A política da sinceridade, aliada com a
solidariedade, cria um ambiente no grupo em que afetividade e luta se
alinham no propósito de promover um encontro potente entre as pessoas.
Apenas quem passa por sua cadeira quente e dos demais membros do grupo
terá condições de compreender e absorver os ganhos da Soma em sua
totalidade.

Depois de realizadas as cadeiras quentes de todos os participantes,
chega"-se a vez do terapeuta. Ele também recebe do grupo as informações
colhidas sobre seu comportamento durante o processo da terapia.
Concluída a última cadeira quente, podemos realizar a terceira maratona
de campo. Quando esta não ocorre, o processo acaba após a última cadeira
quente, que geralmente é a terapeuta. Um ano depois do término do grupo,
ele volta a se reunir com o terapeuta para uma avaliação de como está a
vida de cada ex"-participante. Este encontro chama"-se \emph{Chão Quente},
e são realizados tantos quantos forem de interesse e necessidade do
grupo.

\section{Desenvolvimento do Processo Terapêutico}

A Soma se constitui como um processo terapêutico breve, que não costuma
passar de dezoito meses. Essa é uma condição defendida por nós para
evitar a formação de qualquer tipo de dependência à terapia ou ao
terapeuta. Nosso objetivo é oferecer um processo curto, que se conclua
como um todo e que auxilie as pessoas na construção de sua autonomia.
Mais que promover uma cura absoluta da neurose e dos conflitos
emocionais, uma fabulação tão idealizada quanto falsa, nos interessa
pensar o processo terapêutico como a confecção de uma caixa de
ferramentas para que cada um possa, ao longo e ao final do processo,
criar táticas de uso delas e manejá"-las de maneira singular diante dos
desafios e dilemas da existência.

Visto assim como um todo e dividido apenas para efeito didático, o
processo terapêutico em Soma compreende as seguintes etapas: introdução
pedagógica, socialização, dinâmica, resistência, crise, superação da
crise e cadeira quente. Em todas essas etapas, o efeito diagnóstico anda
junto e simultâneo ao terapêutico. A percepção dos bloqueios que geram a
neurose já é, em si, uma atitude de mudança, um ato terapêutico. E toda
transformação de vida abre novas oportunidades diagnósticas e
terapêuticas. É, portanto, um mecanismo dinâmico e cíclico, não podendo
as partes existir e funcionarem sem as outras. O que nos interessa é que
os membros dos grupos possam transformar percepções e entendimentos em
ações e práticas diante dos mecanismos que bloqueiam suas vidas. E, com
isso, estabelecer a passagem do que podemos chamar de um
autoconhecimento para uma (re)invenção de si.

Na introdução pedagógica, logo no início do processo, oferecemos ao
grupo todas as informações teóricas e vivenciais necessárias ao
entendimento ideológico e terapêutico da Soma. Mostramos como, para nós,
os sintomas neuróticos são decorrentes de bloqueios externos à expressão
livre da originalidade única das pessoas. Entendendo isso, os
participantes do grupo sabem que o nosso trabalho estará voltado para as
causas sociais e políticas que geram os sintomas e não a eles
diretamente. Esta etapa ocorre nos primeiros três meses do grupo, onde
ainda é possível entrar pessoas que irão compor o processo.

Outra característica da Soma é ser uma terapia basicamente lúdica e
socializante. Os exercícios utilizados nas primeiras sessões do grupo já
facilitam esta socialização. Os participantes são estimulados pelo
terapeuta a conhecer um tipo de relacionamento menos hierarquizado,
baseado na autogestão. O desafio de participar de uma interação
diferente, mais sincera e direta, torna visíveis as dificuldades de cada
um para viver sua originalidade e para conviver com as diferenças. A
socialização é o início de um processo de mudança, através da construção
de novas relações sociais libertárias, para a retirada dos saldos da
formação autoritária das pessoas.

A dinâmica de grupo começa a se formar quando já existe um vínculo
afetivo e político entre os membros. Ela estabelece um estado de
colaboração, de cooperação e de cumplicidade nas relações e projetos
entre os participantes do grupo. Sempre produto de ideologia e objetivos
comuns e respeitando"-se as diferenças individuais, uma boa dinâmica
facilita que cada pessoa exerça, da forma mais completa que ela possa,
seus potenciais criativos e produtivos. Baseado nas ideias do
anarquismo, o trabalho coletivo tem mais força e qualidade que o
somatório dos trabalhos individuais. É na dinâmica de grupo que
efetivamente irá se processar a terapia, tanto no plano pessoal
(autoconhecimento) quanto no plano social (novas estratégias de
convivência).

É no conflito existente entre a perda das defesas e dos valores que
caracterizam a neurose e a possibilidade de mudança para uma nova forma
de ser e conviver que costuma surgir a \emph{resistência} à terapia. Ela
é um mecanismo inconsciente de luta contra o processo terapêutico, no
qual o paciente contradiz a intenção e vontade conscientes em provocar
transformações em sua vida. Superar as resistências significa enfrentar
os medos que bloqueiam e impedem os necessários enfrentamentos na
terapia e na vida social.

Podemos distinguir do ponto de vista psicológico a existência de dois
tipos de medos. O primeiro podemos aqui chamar de medo real ou
biológico, aquele responsável por nossa sobrevivência, uma vez que
necessitamos dele para nos proteger de uma situação real e ameaçadora à
nossa sobrevivência enquanto seres vivos. O instinto de sobrevivência
nos permite perceber quando algo está pondo nossa garantia de vida em
jogo, nos alertando com o medo para que possamos tomar uma atitude. Este
``sinal vermelho'' é fundamental e imprescindível para a espécie humana,
não podendo qualquer indivíduo se afastar dele.

O segundo tipo de medo é de ordem neurótica e fantasiosa, ao qual nossas
resistências se aderem. Nesta situação, o medo é produzido desde o
início de nossa infância e caracteriza"-se por seu aspecto fantasioso e
catastrófico, e geralmente associado a um mecanismo político de controle
e repressão. É o chamado medo do medo, aquele que antecipa qualquer ação
antes mesmo da pessoa tentar executá"-la. É sobre este tipo de medo que a
sociedade está cada vez mais amparada, tornando a própria existência
humana mergulhada numa série de temores para além da realidade. Além
deste tipo de medo manifestar"-se em função das transformações que a
terapia produz, também podem ocorrem em situações de vida onde a pessoa
sinta"-se ameaçada e/ou exposta.

Uma pessoa em estado neurótico, por mais que sofra por sua condição,
consegue moldar"-se e buscar um jeito de sobreviver. Sair desse esquema
leva à vivência do novo, à experiência do risco no lugar da segurança
das antigas defesas. É aí que ocorre, por medo da mudança, a
resistência. Ela se manifesta por meio de atitudes bastante variadas,
desde reações neurovegetativas quando a couraça é mobilizada e apresenta
seus sintomas, até rejeição ao grupo, à terapia ou ao terapeuta.

Vencer a resistência é fundamental para se trabalhar com a neurose e
suas defesas, e vai depender da eficiência do processo terapêutico para
se chegar a isso da melhor forma. A luta que se estabelece entre a terapia
e a resistência à terapia acaba levando o grupo e cada um de seus
membros à \emph{crise}. Ao contrário do sentido normalmente usado para
definir a palavra crise, para nós, ela representa algo positivo,
favorável. É através dela que encaramos os obstáculos e impedimentos
para a vida e achamos as possibilidades de solução para vencê"-los.

A superação da crise representa o ``ponto de viragem'' terapêutico, a
partir do qual se é possível passar de tais entendimentos
diagnósticos às ações efetivas de mudança de atitudes no cotidiano,
levando a resultados significativos na terapia. Cada um começa a
entender mais claramente seus mecanismos neuróticos, deslocando a ação
do plano psicológico para o ético"-político, na elaboração de novos modos
de vida. Esta passagem é fundamental para a maturidade necessária ao
grupo no trabalho da cadeira quente.

\section{A arte-luta da capoeira angola}

Procurando ampliar a ação bioenergética durante o processo terapêutico,
no início da década de 1990 introduzimos a prática da capoeira angola à
Soma. Isso ocorreu quando constatamos a importância de um trabalho
corporal mais constante e eficaz para ajudar na descronificação da
couraça muscular e na economia energética necessária para enfrentar os
conflitos neuróticos. Entre as modalidades corporais que tínhamos
disponíveis, a capoeira se mostrou a mais instigante e completa, além de
ser algo ligado às raízes históricas do Brasil. Chegamos a pesquisar
outras modalidades de trabalhos corporais, como o tai chi chuan, a
natação, a dança afro etc., que produzem efeitos semelhantes na
descronificação da couraça. Mas nenhuma delas é tão rápida e eficiente
como a capoeira angola na mobilização da bioenergia e na disposição para
a luta, fator imprescindível no processo de mudança.

Os resultados dos exercícios bioenergéticos são apenas provisórios na
luta permanente que travamos contra os efeitos causados pelo
autoritarismo sobre a subjetividade das pessoas. Eles diminuem a tensão
da musculatura, mas não são suficientes nem garantem a manutenção da
couraça descronificada. Agindo como um forte trabalho neuromuscular, a
capoeira mobiliza praticamente todos os músculos do corpo, liberando a
energia estagnada e auxiliando no restabelecimento da circulação livre
da energia vital.

Como já mostramos na primeira parte deste livro, Reich fez uma
importante descoberta quando localizou sete regiões do corpo onde se
formam as tensões musculares, os chamados anéis ou segmentos de couraça.
Os movimentos da capoeira, com sua ginga e golpes de defesa e ataque,
atuam sobre praticamente todos esses anéis de couraça simultaneamente.
Desde a couraça pélvica, a ``cintura presa'', até a couraça ocular, a
testa ``franzida'', tudo está em constante atividade neuromuscular,
distendendo"-se e contraindo"-se, realizando um movimento constante em
todo o corpo.

Trabalhamos com a capoeira angola, também conhecida como capoeira"-mãe,
numa alusão à sua história e importância. Ela emerge no período da
escravidão do Brasil colonial, quando era praticada dentro das senzalas
ou nas matas, onde o negro disfarçava sua luta em dança, rituais e
brincadeiras. Ali, o escravo preparava"-se para enfrentar a violência e
tirania da escravidão, uma normalidade instituída e perpetuada por
séculos no país.

É comum grande parte das pessoas apenas conhecerem variações atuais da
capoeira angola, com a capoeira regional ou até mesmo o que alguns
definem como capoeira contemporânea. Essas variações aconteceram a
partir da década de 1930, quando um capoeirista baiano chamado Mestre
Bimba criou a Luta Regional Baiana, que depois ficou conhecida como
capoeira regional. Fruto da mistura com outras lutas, a própria capoeira
regional vem sofrendo modificações, tornando"-se mais competitiva e
violenta.

Já a capoeira angola busca preservar as tradições e rituais do passado.
As mandingas e dissimulações caracterizaram sua estratégia de luta, pois
como era uma prática proibida durante muitos anos, necessitava
``disfarçar'' seus elementos de enfrentamento em passos de dança, jogos
e rituais religiosos. Considerado guardião da capoeira angola, o Mestre
Pastinha costumava dizer: ``Capoeira Angola, mandinga de escravo em
ânsia de liberdade. Seu princípio não tem método e seu fim é
inconcebível ao mais sábio capoeirista. Capoeira é amorosa, não é
perversa. É um hábito cortês que criamos dentro de nós, uma coisa
vagabunda.''

A capoeira angola caracteriza"-se por movimentos mais lentos e rasteiros,
que ativam um número maior de músculos, produzindo uma massagem mais
eficiente do ponto de vista neuromuscular. Além disso, ela proporciona
maior conscientização do movimento, importante fator para a percepção
corporal. O jogo da capoeira deve buscar estabelecer um diálogo, numa
comunicação harmônica entre os corpos dos capoeiristas.

A história da capoeira angola é outro importante fator para sua adoção
como instrumento terapêutico em nosso trabalho, pois é parecida com a
história da Soma. Ambas surgiram como práticas de libertação de
situações opressivas, onde as limitações à liberdade eram encaradas como
algo natural. Os negros inventaram a capoeira como forma de resistência
à escravidão imposta pelos brancos no Brasil, transformando seus corpos
em arma de luta. Durante o período da ditadura civil"-militar, a Soma foi
criada como instrumento psicológico para fortalecer a luta pela
dignidade e liberdade.

A luta é elemento básico para o enfrentamento dos mecanismos de poder
que atuam e tentam impedir a auto"-regulação e a liberdade de ser.
Podemos observar que a disposição de luta numa roda de capoeira angola
está relacionada às nossas atitudes de luta na vida. Desta forma, a roda
de capoeira é um treino e um diagnóstico de como estamos lutando e de
como estamos demonstrando nossos enfrentamentos no meio social. Como já
vimos, nosso esquema corporal é um reflexo direto de nova vida emocional
e vice"-versa.

Se uma pessoa mostra no seu cotidiano atitudes de submissão, seu corpo
tende a demonstrar isso através de sua postura. Assim como uma pessoa
autoritária, desconfiada, arrogante etc. Então, ao prepararmos o nosso
corpo para a luta, estamos atuando também sobre as nossas emoções,
adquirindo coragem e confiança na realização de nossos objetivos. É
dessa maneira que a capoeira angola também atua sobre a couraça
somática, facilitando o processo de mudança nas posturas de vida que
assumimos. Durante o processo terapêutico, nossa intenção não é formar
capoeiristas, isso depende do prazer individual e das habilidades de
cada um. O que propomos apenas é que os participantes adquiram um mínimo
de capacidade para a luta que os auxilie nas transformações por mais
liberdade.

É também com a capoeira angola que pretendemos perceber o uso de nossa
agressividade. Como já foi apresentado no primeiro capítulo, é
importante não confundir agressividade com violência. Para nós, ser
agressivo faz parte do ato de viver. A vida é um constante ato de
escolha, de opção, e usar a agressividade de forma natural e equilibrada
significa buscar o que queremos e/ou necessitamos. Cada vez que abrimos
mão da realização dos nossos desejos, estamos bloqueando nossos
impulsos, que mais tarde poderão se transformar em atitudes de compulsão
violenta.

Estimulamos que os participantes de Soma continuem praticando a capoeira
angola como instrumento de trabalho terapêutico mesmo depois de
terminada a terapia. Ela é uma atividade autônoma de manutenção da saúde
para quem a pratica. Depois de vencidas as dificuldades iniciais, como o
preconceito e o medo do risco, a capoeira acaba virando sinônimo de
prazer.

As cantigas, o toque do berimbau, a mandinga, os movimentos, a roda,
tudo na capoeira tem um aspecto lúdico. Por isso não se \emph{luta}
capoeira, \emph{joga"-se} capoeira. Ela é um jogo, em que a malícia e a
habilidade determinarão a estratégia. Na roda, joga"-se com todo o corpo,
onde os movimentos buscam ser harmônicos e comunicativos. As pessoas que
praticam a capoeira são envolvidas de uma forma que a emoção, a razão e
o físico integram"-se plenamente. Mais do que uma simples dança ou luta,
a capoeira angola é um estilo de vida, uma maneira de encarar o ato de
viver.

Nas sessões terapêuticas da Soma, algumas delas são destinadas à prática
da capoeira angola, com o próprio somaterapeuta ou eventualmente com
mestres e capoeiristas de fora que são convidados a oferecer oficinas a
partir de diferentes aspectos da capoeira, que abrangem desde a
musicalidade até movimentos e rodas. Nestas sessões, após exercícios de
aquecimento próprios, realizam"-se as \emph{rodas de capoeira}, nas quais,
como nos outros exercícios de Soma, surgem novas ``bandeiras''. Da mesma
forma que nas outras sessões, é feita a leitura e o fechamento. Nesse
caso, a capoeira angola serve como referencial, tanto para estudo
bioenergético quanto de capacitação para luta.

Apesar de ser fundamental para a Soma, o trabalho realizado nas sessões
de terapia com a capoeira é insuficiente para desenvolver os potenciais
de quem a pratica. Por isso, recomendamos ao grupo a prática da capoeira
em paralelo às sessões de terapia. Isso depende muito do
interesse e disponibilidade de cada um. Mas acreditamos que duas vezes
por semana seja uma quantidade suficiente para qualquer pessoa se
qualificar e obter dela seus benefícios. Nas cidades onde
trabalhamos, costumamos ter vínculos de parceria com grupos e
capoeiristas que oferecem treinos regulares.

\chapter*{Considerações sobre uma psicologia libertária}
\addcontentsline{toc}{chapter}{IV. Considerações sobre uma psicologia libertária}

A Soma tem mais de quarenta anos de atividades no Brasil e na Europa.
Apesar das mudanças que vem experimentando ao longo dos anos, parte
substancial de suas bases teóricas e de suas práticas surgiram das
pesquisas originais de Roberto Freire, fruto de suas incursões em
teatro, psicologia e ação política. Algumas vezes acompanhado por
cúmplices que lhe deram apoio, muitas outras sozinho, Freire criou esta
potente abordagem eminentemente brasileira. Nosso desafio atual tem sido
em contribuir e ampliar a prática da Soma, através da criação de novos
exercícios, na articulação com novos saberes ou mesmo na expansão de sua
aplicabilidade.

Sendo uma obra aberta, a Somaterapia mantém"-se em movimento, e por
caracterizar"-se enquanto processo terapêutico"-pedagógico libertário está
envolvida na permanente relação entre o comportamento individual e a
construção de sociabilidades livres e não hierarquizadas. Desta forma,
sua realização e desenvolvimento sempre buscou, a partir da prática da
autogestão e da solidariedade mútua, uma maneira própria de organização
que propicie esta permanente construção frente às transformações pelas quais
passam o indivíduo e a sociedade.

Foi assim que no início da década de 1990, reunidos na Ilhabela, litoral
sul de São Paulo, traçamos as bases para criação de um veículo capaz de
dar continuidade às pesquisas de Freire, desenvolvendo e difundindo a
Soma através de livros, cursos, grupos de terapia etc. Surgia o Coletivo
Anarquista Brancaleone, reunindo Roberto Freire e uma equipe de
somaterapeutas ligados a ele. Inspirados pelo clássico do cinema
italiano \emph{L'Armata de Brancaleone}, de Mario Monicelli, uma sátira
bem humorada sobre a necessidade da utopia, criamos o nosso Coletivo
Brancaleone como veículo de ação libertária tanto para a prática da
Somaterapia, como para intervenções dentro do meio social.

O Brancaleone de Monicelli foi um importante símbolo de resistência
política dos jovens envolvidos com as lutas sociais e ideológicas das
décadas de 1960 e 1970, no Brasil e no mundo. Inspirados pelas lutas e
aventuras atrapalhadas do cavaleiro Brancaleone, a comédia italiana
representava àqueles jovens a possibilidade de fazer política
distante da sisudez dos partidos políticos marxistas. Eles reivindicavam
o lúdico, o humor e a utopia como ingredientes centrais da militância
revolucionária tão longe das tradicionais formas de ação política.

Ícone de seu tempo, o filme \emph{L'Armata de Brancaleone} também nos
serviu de inspiração anos depois de seu lançamento. De alguma forma, a
trajetória da Soma ao longo desses anos não deixa de ser uma luta
``brancaleônica''. Afirmo isto por considerar que depois de tantos anos
de existência, foram poucos somaterapeutas que formamos. Sempre em
pequeno número, porém lutando e acreditando juntos nas possibilidades do
anarquismo contemporâneo, especialmente no que chamamos de ``anarquismo
somático''. Atuando em coletivo, através de autogestão, nosso
Brancaleone em muitos momentos tentou desfazer a lógica suicida do
conformismo normalizado da sociedade em que vivemos, que não vê saídas
que não sejam as mesmas de sempre, à frente da realidade que se apresenta
diante de nós.

Mesmo com a morte do Roberto, em maio de 2008, o Brancaleone manteve"-se
reunindo os somaterapeutas em atividade ou em estágio de formação. Nos
últimos anos, alguns companheiros surgiram e tantos outros se foram,
fruto da luta constante por uma postura ética que fosse compatível com
nossas paixões e desejos. Desde a primeira formação, o somaterapeuta
João da Mata manteve"-se no Brancaleone, dando continuidade ao trabalho %É assim mesmo, o autor se referindo a si na terceira pessoa?
original de Freire. Hoje seu desafio tem sido o de continuar as
pesquisas da Somaterapia, nos campos terapêutico e pedagógico, e formar
novos terapeutas. Ao Brancaleone, sua tarefa atual é de recriar"-se e
reinventar"-se, para que possa seguir como um dos núcleos de pesquisa da
Somaterapia.

\section{O anarquismo somático e a terapia}

Pela importância que representa para a Soma, gostaria de deixar
registrado algumas reflexões em torno do pensamento libertário e sua
especificidade a partir do que chamamos de \emph{anarquismo somático}. Suas
características estão ligadas à defesa do prazer como valor ético, ao
entendimento do corpo como unidade indivisível, à atuação libertária no
aqui"-e"-agora e ao entendimento do ser humano dentro de uma perspectiva
bio"-psico"-social. Ao procurar ir além da divisão política apoiada no
arco que corresponde à esquerda e à direita, ambas operando na
perspectiva do Estado e na busca de tomada do poder, e propor uma nova
ação política, vemos o anarquismo como a ideologia do prazer que combate
e resiste à ideologia do sacrifício, amplamente difundida não só pelo
neoliberalismo globalizado, mas também pelos cânones do marxismo e da
psicanálise, duas teorias arraigadas no imaginário do senso comum.

O anarquismo somático estabelece uma crítica ao ideal arcaico de
revolução social, entendido através da tomada do poder e da implantação
de qualquer outro, mesmo que este se afirme como libertário. Acreditamos
que a atitude libertária se dá no aqui"-e"-agora, em experiências que
buscam relações horizontais, combatendo hierarquias que se estabeleçam
enquanto jogos de poder. A Soma enquanto terapia anarquista defende a
noção de saúde atrelada a uma dimensão ético"-estético"-política e, para
tal, entende ser necessária a construção de relações apoiadas numa
sociabilidade não"-hierarquizada, construída no exercício prático da
autonomia. Acreditamos que as diferentes práticas de poder e dominação,
sejam elas econômicas, morais, de gênero e raça, todas elas, operam e
produzem efeitos significativos na saúde emocional das pessoas.

Ao utilizar uma metodologia que privilegia a reflexão sobre os jogos de
poder, o anarquismo somático está presente no processo da Somaterapia
através de uma dinâmica de grupo autogestionária, na valorização das
diferenças e na quebra das hierarquias. Nos grupos de terapia, a
metodologia autogestiva produz uma análise crítica dos indivíduos que
compõem esse micro"-laboratório social. Esta aposta no anarquismo ligado
ao dia a dia, vivificado na esfera individual e coletiva, especialmente
através da autogestão, é o caminho encontrado por nós para a construção
e a aplicação deste processo terapêutico.

Ao referir"-se ao pensamento libertário presente na Soma, Roberto Freire
costumava dizer que ele representava uma forma contemporânea e original
de socialismo libertário que, trabalhando o indivíduo em
microssociedades experimentais (grupos terapêuticos), leva"-o a
revolucionar sua microssociedade espontânea (acasalamento, família,
amizades, colegas de trabalho). Fundamentalmente, a pessoa que faz Soma
vai aprender a viver as pulsões de seu corpo, de seus relacionamentos
afetivos, a nova organização familiar, suas inéditas relações de
trabalho de forma autogestiva e libertária.

As lutas e ações dos anarquismos sempre estiveram interessadas na
formulação de práticas políticas no dia a dia, distantes da burocracia
dos partidos e seus políticos profissionais. Uma luta da vida cotidiana
em associações que buscam inventar jeitos livres para amar, criar e
produzir. Em acontecimentos históricos ou atuais, os anarquismos
procuram criar sociabilidades sem a égide do Estado, em acontecimentos
heterotópicos nos quais as liberdades individuais e coletivas buscam
articulações sem hegemonia de umas sobre as outras.

A noção de heterotopia busca dar conta da efetivação de espaços de
liberdade no presente. Contrária à noção clássica de utopia, que nos
remete ao futuro distante, no qual lá na frente seremos contemplados por
uma vida livre e satisfatória, a heterotopia encarna no aqui"-e"-agora
a construção de pactos de autonomia. O conceito de heterotopia emerge
pela primeira vez na obra do filósofo Michel Foucault em \emph{As
Palavras e as Coisas} (1966), quando é examinado apenas em relação ao
discurso e à linguagem. No ano seguinte, em 1967, em artigo de poucas
páginas escrito na Tunísia, e posteriormente publicado nos anos de 1980,
chamado \emph{Outros Espaços}, Foucault retorna o conceito, agora lhe
ampliando o sentido para um referencial material. Neste artigo, o
filósofo está interessado em formular um conceito que carregue a ideia
de espaços de invenção e resistência no presente. O autor procura, com
isso, romper o sentido de lugares situados no futuro, como espaço
privilegiado a ser atingido. Para Michel Foucault, as heterotopias são
``lugares reais, lugares efetivos, lugares que são delineados na própria
instituição da sociedade e que são espécies de contraposicionamentos,
espécies de utopias efetivamente realizadas nas quais os posicionamentos
reais, todos os outros posicionamentos reais que se podem encontrar no
interior da cultura estão ao mesmo tempo representados, contestados e
invertidos, espécies de lugares que estão fora de todos os lugares,
embora eles sejam efetivamente localizáveis'' (\versal{FOUCAULT}, 2003, p. 415).
Ou seja, as heterotopias são espécies de utopias possíveis, no sentido
em que se constituem como espaços reais, localizáveis e atuais, mas cuja
característica é a de serem, constitutivamente, outros espaços.

Esta noção nos parece pertinente para pensar os grupos de Soma. Eles se
constituem como espaços outros, transitórios e móveis, nos quais podemos
experimentar práticas livres, seja por meio da fala franca, do apoio
mútuo, do exercício agonístico das diferenças e de uma série de outras
ações. Possivelmente um dos mais importantes diferenciais da Soma
enquanto processo terapêutico reichiano e em grupo seja a possibilidade
de conhecer uma sociabilidade nova, que produza redimensionamentos
diante daquelas que são ofertadas ao longo de nossas vidas.

Através do anarquismo somático que permeia o processo da Soma,
permanecemos amigos em associações quando respeitamos as diferenças de
cada um e buscamos relações sem hierarquias, descobrindo o prazer da
produção coletiva na busca da unidade na diversidade. Assim, no decorrer
de um grupo de Soma, muitas vezes sofremos com as separações de pessoas
que abandonam o processo, seja por resistências à terapia, mudanças de
vida ou qualquer outro motivo. Mas constatamos também sua
inevitabilidade para continuarmos trabalhando em autogestão.

No alto das possibilidades das virtudes libertárias, a amizade é eleita
como a mais soberana e afirmativa das formas de relação com o outro. Ela
é eletiva, na medida em que se dá por livre associação, num encontro que
passa ao lado do jogo social. Nos coletivos que representam os grupos de
Soma, ela costuma emergir numa relação de pessoas concordantes por
escolha mútua, sempre provida de uma carga de afetividade. Fundada na
cumplicidade, a amizade tende a tornar"-se a justa medida do exercício da
troca anarquista: a virtude sublime por excelência.

E é esta amizade, com gosto de cumplicidade, que costuma manter coesos
os membros dos grupos da Soma. E foi ela também que me fez estar ao lado
de Freire nos últimos vinte anos de sua existência. Mesmo no fim da
vida, já com muitos problemas de saúde, Roberto Freire ainda militou
como amante apaixonado. Ele, mesmo com todos os motivos para se deixar
levar pelo cansaço da velhice, não parou de inventar amizade e
cumplicidade nem um só instante.

Em tempos de controles e monitoramentos das condutas, é cada vez mais
urgente sair das caixas e moldes para expandir nossas ações e relações.
Cada um pode e deve governar a si mesmo através das práticas de
subjetivação e dos modos de constituição de si como sujeitos, em
acontecimentos que ocorrem junto ao outro como contínua e incessante
relação. Nesta associação, a disputa para ser livre não ocorre com a
anulação do diferente, mas ao contrário, ao valorizar a singularidade do
outro é que construímos nossa existência. Este desafio situa"-se também
na lucidez e na certeza que o campo de batalha nunca acaba por completo.
Na elaboração de si para a vida livre, podemos contar com aqueles que se
situam nos círculos das amizades para aumentar nossas potências e
produzir paixões alegres.

Tem sido assim que tenho vivido estas quase três décadas de atividades,
convivência e produção na condução da Soma. Nos encontros com tantas
pessoas e diferentes grupos, a autogestão e amizade libertária nos une
enquanto produzimos trocas, atravessados por movimentos de liberdade,
criatividade, solidariedade e amor. Este desejo de continuar acreditando
no anarquismo cotidiano, vivificado no dia a dia, me mantém ativo em
praticar a Soma com luta e tesão pelo mundo afora. E seguir acreditando
na utopia e na paixão vividas no presente.

\section{Erotismo, sensualidade e sexualidade como~potências~da~vida}

Foi Wilhelm Reich, o mais radical dissidente da psicanálise freudiana,
que trouxe a importância do exercício pleno da sexualidade como arma
revolucionária, capaz de promover uma intensificação da vida. Apesar de
hoje assistirmos a uma proliferação de temas relacionada ao sexo e mesmo
a um amplo comércio em torno dele, parece que estamos distantes da
revolução sexual que Reich defendia. Para Reich, os elementos envolvidos
nos processos de adoecimento, tanto físico quanto emocional, estavam
relacionados à presença de práticas e discursos moralizadores e suas
consequências antilibidinais. Segundo ele, a psicologia deveria andar
junto com uma análise crítica das relações de poder, a fim de romper com
práticas autoritárias presentes nas religiões, escolas e famílias para
promover uma revolução social e sexual, simultaneamente.

Sua psicologia vai em direção de uma política sexual libertária,
implicada na criação de sociabilidades menos disciplinadoras. Em uma
conhecida afirmação, Reich diz que ``a família espelha e reproduz a
estrutura de Estado'', produzindo ali relações de dominação que vão, aos
poucos, limitando a capacidade de viver com mais liberdade e poder
crítico. Os casamentos entre homem e mulher --- baseados na posse, no
machismo e em relações monogâmicas compulsórias --- assim como a noção de
obediência aos pais criam uma estrutura vertical entre os membros
implicados. A partir do núcleo familiar, estende"-se, segundo ele, uma
série de outras práticas sociais, nas quais a hierarquia é entendida
como condição natural. Parte de sua crítica à família, assim como às
religiões e escolas, ancorava"-se no fato de que estes espaços incidem
sobre a sexualidade livre das crianças e jovens. Suas práticas
disciplinares atuam não apenas sobre o comportamento sexual, mas também
sobre a curiosidade intelectual, a criatividade e a espontaneidade.
Seria justamente aí, através de processos disciplinares e de controle,
que surgiriam indivíduos obedientes ao princípio de autoridade.

A mistificação em torno da autoridade encontra assim terreno fértil para
instauração de políticas fascistas. O que interessa mesmo na perspectiva
de Wilhelm Reich é uma intensificação da potência da vida, por meio de
um vitalismo que perpassa o corpo e a intersubjetividade. Sua aposta
reside no exercício da sexualidade como produtora de saúde. No entanto,
via uma diferença na relação sexual como fusão junto ao outro e a mera
realização do coito, já que sexualizar excessivamente a vida poderia
criar uma perigosa falácia de liberação. Para Reich, a afetividade seria
capaz de garantir uma entrega emocional, que por sua vez seria
responsável por uma dissolução circunstancial do ego, e que provoca uma
entrega plena ao outro. Neste sentido, a sexualidade é bem mais ampla
que o ato sexual em si, mas está envolvida com a própria vivência
cotidiana do erotismo.

A questão é que hoje não se distingue mais o erotismo propriamente dito
e a pornografia, algumas vezes tornando"-a uma deturpação da noção de
erotismo. Enquanto a experiência erótica está relacionada a tudo que diz
respeito ao plano das sensações corporais, a pornografia alimenta"-se das
fantasias e imagens racionais. Vivemos o erotismo hoje em termos de
consciência e limitado ao relacionamento sexual. Daí supõe"-se que o
erotismo só serve para a prática do sexo. Este esvaziamento do erotismo
o leva a ser apenas um instrumento para a efetivação do ato sexual.

Erotismo e sensualidade, apesar de quase sempre estarem relacionas ao
ato sexual, estão também para além dele. Uma existência sensualista diz
respeito ao mundo das sensações, que percorre o corpo afetado pelos
sentidos. Assim, é possível sentir uma certa dose de sensualidade no ato
de comer uma boa comida, de escutar uma música, em contemplar a beleza.
Ou seja, a sensualidade está relaciona a todo prazer que afeta o corpo.
Uma existência erótica, percorrida de sensualidade, não precisa apenas
do ato sexual para manifestar"-se. Ela está presente nos pequenos gestos,
no dia a dia, intensificando a vida, percorrida por energias e
vitalismo. O que torna o erotismo algo pornográfico é fruto da mesma
moral conservadora, base da cultura patriarcal, que procura suprimir os
impulsos sensuais e naturais. Isso leva à criação de impulsos
secundários, tornando a sexualidade uma mera realização do coito, muitas
vezes desprovida de qualquer sentimento. Ao mesmo tempo, ergue"-se um
conjunto de leis e normas moralistas desastrosas contra a mesma mente
humana pornográfica criada pela repressão da sexualidade natural.

Para Reich, a neurose se constitui durante as principais fases da vida:
primeira infância, adolescência e idade adulta. Os bebês e as crianças
convivem com uma atmosfera familiar muitas vezes neurótica, autoritária
e conservadora do ponto de vista sensual e sexual. As exigências dos
pais em relação à boa conduta, ao bom comportamento e às
autorrestrições, produzem o que na puberdade se configura como
desinformação e desconhecimento para uma vida sexual satisfatória. Por
fim, na idade adulta, a maioria das pessoas se vê envolvida na armadilha
do casamento compulsório, muitas vezes desprovido de um real prazer
afetivo e sexual.

Em \emph{A Função do} \emph{Orgasmo}, Reich mostra como o orgasmo sexual
pleno, além de proporcionar grande prazer, tem uma segunda função capaz
de produzir uma poderosa descarga energética, que dissolve as tensões
musculares e restitui circunstancialmente o equilíbrio da energia vital.
Isto seria alcançado graças ao que Reich chamou de ``potência
orgástica''. Ao relacionar a neurose às perturbações da função genital,
a atividade orgástica passa a ter um significado importante no
tratamento e elaboração da saúde somática e psíquica de seus pacientes.
Se a ênfase colocada por Reich na economia sexual de seus pacientes pode
significar uma simplificação das questões envolvendo o adoecimento,
serve para alertar as diferenças entre casamentos como contrato social e
econômico ou uma associação pautada no desejo. Além do mais, Reich vai
estender a noção de orgasmo para além do sexo, ampliando seu conceito
para diferentes áreas da vida. Ou seja, ter orgasmos significa
entregar"-se à plenitude e abandonar"-se às experiências: deixar de ser
governado.

E neste sentido, a construção de uma vida erótica está apoiada nas
experiências diárias de nossos tesões: naquilo que nos traz prazer,
beleza e alegria. Para além do tesão como desejo sexual, esta potência
emerge nas amizades, nas relações entre pais e filhos, na relação com o
trabalho e a criação, enfim, nos inúmeros instantes que fazem da vida
acontecimentos imanentes e afirmativos. Possivelmente, a revolução
sexual que Wilhelm Reich defendia estava relacionada a esta erotização
do cotidiano.

\section{Por uma erótica solar}

Ainda vivemos com demasiada intensidade um corpo platônico, cindido e
esquizofrênico. Cortando em duas partes, a hegemonia do pensamento sobre
as sensações continua a reinar fortemente, e sua incidência começa cedo
e continua pela vida: em casa, nas escolas, na lógica do trabalho e na
produção do dinheiro. A tradição hegemônica do dualismo platônico,
radicalizado e pulverizado graças ao esforço do cristianismo, tratou de
apontar o corpo como o local do erro. A alma tomada como superior,
eterna e imutável, adquire importância e supremacia sobre as sensações
corporais, pois delas advém o ``desvio moral''.

A vitória do platonismo encontra nas religiões monoteístas em geral,
fortes aliadas voltadas ao empenho de afastar do corpo a possibilidade
de extrair sentidos da própria existência. Este legado representa a
vitória de uma visão de mundo idealizada, presa ao campo do imaginário.
O platonismo estabelece as bases do abandono do corpo, o desprezo pela
carne e pela matéria, e a consequente valorização da alma. Por
conseguinte, o discurso apaziguador e dócil das religiões esconde o
verdadeiro objetivo de sua moral: a restrição à liberdade e à vivência
do prazer. Surge ainda a ideia do corpo como o \emph{locus} do pecado,
que passa a partir daí a criar uma série de mistificações sobre o
desejo, a sexualidade, os homossexuais, as mulheres e o livre
pensamento.

As sequelas, e sob as quais a maior parte das pessoas vive ainda hoje o
cotidiano de seus corpos, são bastante conhecidas: culpabilidade,
temores, medo, angústia, revolta contra si próprio, sentido de
perversidade e desvalorização da carne. Na espreita desse pensamento,
fabricam"-se a castidade, a virgindade e a vergonha do prazer, para então
desaguar no casamento como contrato social, essa sinistra combinação
gregária e seu erotismo de rebanho. Pensar, portanto, uma ética voltada
à eleição do prazer, significa confrontar"-se contra esta tradição, assim
como voltar"-se para uma filosofia do corpo, atéia e sensualista, que
busque combater modos de vida assujeitados e conservadores.

Querer um corpo não cindido, pagão e sensualista é inscrever"-se no mundo
de forma radicalmente contrária ao ideal ascético, a partir de uma
dinâmica que se faz mediante uma energia de que o corpo é portador. A
reconciliação com a corporeidade passa inevitavelmente por acabar com os
mundos remotos, os céus e os supostos lugares por onde habitam as ideias
e as essências. A afirmação de uma erótica solar, ao romper com esta
tradição, inscreve a matéria como a própria ``instância'' onde se
origina e se exerce o prazer. Energia que percorre o corpo, movida pelos
sentidos, para enfim, produzir nossa inscrição no mundo. A erótica solar
é praticada nas carnes vivas, movidas pelo desejo que percorre o corpo
em sensações imanentes, distantes de idealismos e transcendências.
Desejar deriva da necessidade de uma dinâmica fisiológica, assim como de
uma imanência corporal.

O filósofo francês Michel Onfray, na elaboração do que chamou de
materialismo hedonista, defende o prazer como virtude ética e elabora
uma genealogia possível sobre a moral cristã, considerada por ele como
``uma máquina de fazer anjos''. Para tal, percorre uma galeria de
devassos, cujo trajeto hedonista revela a história de homens e mulheres
que não compreendiam a possibilidade de modos de vida sem seus
cruzamentos com a insubmissão e o prazer. Homens e mulheres para quem os
instantes fugidios da sabedoria aconteciam em co"-extensão com os
instantes culminantes de rebeldia, satisfação e gozo.

O propósito o qual defende Onfray em seu materialismo hedonista é de
colocar"-se contra o que julga ser a hipocrisia, a moral moralista, a
ideia do pecado e do medo que fazem do corpo e do prazer algo que se
deva ter aversão e horror. Por que culpabilizar o desejo e a
sensualidade, por exemplo, e não a fome e o descanso? São todas elas
sensações percorridas pela matéria. São as morais ascéticas que
transformaram o prazer e o gozo em algo que se deva ter vergonha, ser
escondido pelo medo da crítica social. Mas na verdade, são sensações
fisiológicas que estão para além da classificação de bem ou mal.

Nas filosofias materialistas, a fisiologia está a serviço da liberdade,
entendida como a capacidade de exercer a autonomia a partir das
informações obtidas no contato com a realidade e das impressões
corporais advindas daí. Seguimos na direção daqueles que procuram
afirmar um certo monismo filosófico que conduz à invenção do corpo uno e
material, radicalmente imanente, que nos forneça informações do mundo a
partir do contato direto com ele, distante de qualquer verdade que se
coloque como ideia em si. A experiência e a sensação são tomadas como
premissas para o acesso ao acontecimento. Corpo em movimento, carne
percorrida por energias agradáveis e distantes daquelas desagradáveis: a
apropriação da corporeidade produz uma sabedoria do organismo.

Contra o corpo esquizofrênico, comum naqueles que vivem modos de vida
acomodados e temerosos, acreditamos que não deva haver depreciação da
carne. O dualismo, a alma imaterial, a transcendência ou um além"-mundo
fazem parte de uma tradição tão forte quanto ficcional. É apenas no aqui"-e"-agora, no mais puro encontro com o real, que afirmamos nossa
existência atomista: a matéria, a vida, o vivo. Uma erótica solar rompe
com a lógica que situa o desejo como carência, para então afirmá"-lo como
transbordamento. O prazer não se define pela completude, mas na
conjugação que traz o excesso e a demasia. Sendo descarga, derramamento,
o desejo na erótica solar pressupõe excesso, dispêndio de energia e
realização do corpo. Aqui, em substituição à noção de algo completar na %Completar mesmo?
busca do prazer e toda a mística que se cria em torno dela, emerge o
suplemento.

A aposta em uma cultura erótica que entende o prazer como potência se
ocupa em produzir efeitos estéticos e júbilos na existência. Como
contraponto a uma libido melancólica, a erótica solar é o prazer sem
culpa, sem medo. Mais que isso: é o próprio vitalismo que anima a vida.
Uma perspectiva ética, que também é estética e tem como objetivo
contraposicionamentos e práticas de resistência às morais moralistas.
Nas relações libertárias, pautadas no desejo mútuo e na autonomia entre
os envolvidos, a erótica solar é vivida entre pessoas que buscam
conservar as prerrogativas e o uso de sua liberdade. Ela advém do
erotismo que pauta o desejo encarnado, despudorado e inventivo.

Aliás, (re)criar"-se nas relações que podemos chamar de amorosas faz
parte do espírito livre e rebelde que só aceita estar com outro pela
livre associação. Sempre que possível vale à pena nos questionarmos como
vivemos nossos amores e o exercício de nosso prazer. A diferença entre
uma relação sexual duradoura e o estabelecimento de casamento
compulsório é grande. As armadilhas da dependência, do medo e da
alienação de si criam um sutil verniz que ofusca a liberdade e a
vivência do prazer.

Isso me faz lembrar o libertário e iconoclasta Roberto Freire, quando
escreveu em seu \emph{Ame e dê Vexame} o que chamou de declaração do amante
anarquista: ``Porque eu te amo, tu não precisas de mim. Porque tu me
amas, eu não preciso de ti. No amor, jamais nos deixamos completar.
Somos, um para o outro, deliciosamente desnecessários.'' Esta é a aposta
de uma erótica alegre e solar, que é potência selvagem e sem lei.

\section{O intolerável e a vida livre}

Já há algum tempo a palavra \emph{terapia} não é a melhor forma de
definir a Soma. Ao menos, a maneira como este termo normalmente é
utilizado, remetendo a uma noção de cura. Não acredito que os conflitos
humanos possam ser curados, até porque vivemos em sociedades adoecidas,
que retroalimentam processos de controle das mais diversas formas, com
consequências diretas na vida emocional das pessoas. O que penso ser
possível, e isso é fundamental, é cada um conseguir desenvolver táticas
e manejos nos mais diferentes espaços da malha social, capazes de lidar
com as práticas de poder, de forma estas que possam ``respingar'' o
menos possível em nossa vida emocional. Fomentar esse \emph{jogo de
cintura} talvez seja uma das mais importantes e potentes formas de estar
no mundo. E a capoeira angola mostra isso de maneira magistral.

Dessa forma, penso ser a Soma mais um processo experimental --- através do
qual buscamos perceber e ativar potências criadoras, afetivas e
políticas capazes de favorecer cada um na elaboração de vidas
afirmativas --- que propriamente uma psicoterapia no sentido usual. Seu
papel, mais que eliminar os conflitos emocionais por completo, é de
fornecer instrumentos: apresentar uma caixa de ferramentas que são
disponibilizadas ao longo do processo do grupo e que, ao final, cada
participante as tenha e possa fazer o melhor uso delas. Além de romper
com esse fetichismo autoritário de que o \emph{terapeuta} tem algum
poder de curar as pessoas, valorizamos o percurso autoral e a autonomia
de cada um no manejo de sua existência. O processo terapêutico serve,
dessa forma, para incrementar as possibilidades de luta individual e
coletiva capazes de enfrentar os mecanismos neurotizantes disseminados na
sociedade.

Pensar a vida como percurso inventivo, que nos leve também a uma
existência afirmativa, é pensá"-la como campo de batalha. Não se criam
sentidos jubilosos se não há revolta e insurgência diante do que nos
mediocriza. Assim, revoltar"-se significa confrontar"-se com tudo aquilo
que faz diminuir ou arruinar a condição humana. E são muitos os
fatores que operam cotidianamente neste sentido: a ação autoritária de
Estados, a miséria gerada pelo capitalismo, o racismo estrutural, o
machismo e a misoginia, e tantos outros. Sabemos que a lista é enorme e
sua engrenagem retroalimenta"-se na conhecida dinâmica dominador"-dominado
ativadas mutualmente. São muitos os agentes que nos produzem incômodos,
e todos eles, quase sempre, andam juntos com uma leva de acomodados.

Seria fácil, uma vez percebida a fonte do intolerável, afirmar sua
negação e, uma vez constatada a ausência ou perda de sentidos para a
própria vida, que cada um agisse de forma a eliminar o incômodo. Assim
seria, mas não é. A neurose nos condena à condição existencial tão
terrível, que suportamos, feito bestas de cargas, situações
insuportáveis. Criam"-se leis, tecnologias de gestão de tempo e vida,
dispositivos dos mais variados e capazes de administrar nossas vidas de
maneira insípida. Suportar uma vida sem sentido passa a ser algo, em
certa medida, administrável.

Ainda na década de 1930, Wilhelm Reich denunciava que a revolução russa
estaria fadada ao fracasso, pois o proletário soviético vivia o que
chamou de \emph{miséria emocional}. Ele acreditou que a experiência
marxista poderia oferecer um contraponto ao capitalismo e seus efeitos
subjetivos, mas o que viu foi que a ditadura do proletariado havia
criado uma classe dirigente autoritária e que a população não gozava da
necessária liberdade como condição humana de vida digna. Para Reich, não
era possível criar uma nova organização social se os indivíduos não
desfrutassem da liberdade e do prazer como potência de vida. A miséria
emocional cria nas pessoas medo e necessidade de serem governadas.
Acovardadas e carentes de um salvador, geralmente legitimam líderes
autoritários, reverenciam um ídolo, seguem dogmas, defendem raivosamente
conceitos e preconceitos. Temorosas, não se colocam de frente nas suas
críticas, preferem as sombras, o anonimato e a fofoca.

Reich seguia, no âmbito da psicologia, inquietação semelhante àquela
descrita pelo jovem Étienne de La Boétie quando dirigia seu espanto não
ao tirano, mas aos que aceitavam voluntariamente seu poder. Suas análises
se dirigem ao soberano que se coloca como
ponto de partida para as relações de hierarquia. Mesmo o soberano sendo
rei, ditador ou representante do povo, deve"-se obediência a ele,
independente de ser melhor ou mais legítimo. No entanto, qualquer um que
venha a governar condutas será sempre um tirano. Resta saber quando o
tirano se funda a partir de nossa própria servidão voluntária. Nas
políticas de rebanho, dominadores e dominados convivem no diapasão amor
e temor: deveres e obrigações, intercalados por ganhos de ambos os lados.

As análises de La Boétie em seu \emph{Discurso da Servidão Voluntária}
trazem a lucidez de que todo poder se exerce com o assentimento daqueles
sobre os quais se manifesta. Pensar estratégias de resistências passa
necessariamente por esta constatação, para então localizar nossa própria
alienação, seu funcionamento e sua trama. O que Reich busca compreender,
sem dirigir"-se diretamente a La Boétie, é como a neurose estava
relacionada à ideia da aceitação do poder. Para ele, o conflito
emocional se produz através da incitação à obediência, que começa em
casa a partir de uma educação orientada para o consentimento da
centralidade e da hierarquia. De início, a obediência aos pais; depois
aos professores; aos patrões; e por fim, ao Estado, seja ele
representado pela polícia ou tribunais. Para La Boétie, e em certa
medida também para Reich, o espanto com os efeitos danosos do
autoritarismo se dirige mais aos que obedecem do que a quem os produz.

Neste sentido, acreditamos que viver um processo terapêutico passe,
necessariamente, por confrontar"-se com as condições que produzem
sujeições para criar sociabilidades entre livres. Mas se é difícil
enfrentar o poder que opera sobre nós, mais difícil ainda é abrir mão do
poder que agimos sobre os outros. A aparente inevitabilidade de
sociabilidades hierarquizadas tornam as práticas de poder e sujeição
algo comum e banal na existência humana. Romper este ciclo vicioso, com
seu ganhos e vantagens, significa literalmente abrir mão das práticas
intoleráveis que acabam por tornar"-se aceitáveis no jogo social.

Se não confrontamos o intolerável nas mais variadas formas de
autoritarismos, especialmente as mais sutis, acabamos por ficar
acomodados diante dos fatos. No chamado \emph{último Foucault}, o %Caixa-baixa mesmo?
filósofo francês nos alerta para uma das características mais marcantes
da moral burguesa: a acomodação. Resumidamente, Michel Foucault assinala
a acomodação como uma escolha estratégica em manter uma certa
``ignorância'', em conservar valores e acima de tudo, em viver uma
existência de pouco risco e com as garantias da segurança. Delega"-se a
\emph{outrem} a tarefa de governar e guiar subjetividades em que a
acomodação torna"-se mesmo um processo de normalização.

Em suas análises, Foucault contrapõe a acomodação ao modo de vida
artista: possibilidade, através da qual construímos nossos caminhos,
elaboramos nossas vidas de maneira autoral. O modo de vida artista não
diz respeito ao percurso artístico de um determinado artista, mas um
modo de vida libertário singular que cada um pode e deve criar, desde
que deseje produzir sentidos em sua vida, para então elaborar"-se como
obra de arte. Representa também um modo de resistência e luta pela
autonomia, com o objetivo de afirmar contraposicionamentos diante das
técnicas de controle e padrões de normalização.

Roberto Freire, na criação da Soma e na defesa de um processo
terapêutico transformador, trazia insistentemente a noção do
\emph{tesão} como potência revolucionária. A afirmação do tesão tem,
inegavelmente, uma força política. Pode"-se perceber como é na perda do
prazer em que se apoia o autoritarismo. Os poderes autoritários sugerem
uma sujeição da vida humana, a de sua ludicidade. A necessidade de poder
corresponde a uma impossibilidade de se viver os prazeres relativos da
existência cotidiana. A liberdade se produz, na visão de Freire, como um
prazer de estar vivo. Essa seria uma grande arma que dispomos para lutar
contra as limitações que impedem as mutações existenciais individuais e
coletivas. Assim, Freire defendia um olhar crítico sobre a necessidade
de poder em nós: o poder como aquilo que também pode levar a mecanismos
patológicos que nos impedem de viver criativamente, e de inventar uma
nova existência, mais prazerosa, brincante, e ligada àquilo que nos
desperta paixão pela vida. E seguir a vida naquilo que Deleuze chama de
devir revolucionário, incessante e contínuo.

\chapter{Nota}

Este livro representa um primeiro contato com a teoria e prática da
Somaterapia, com o objetivo de ampliar a comunicação e informação das
pessoas que se interessam por nosso trabalho. Nos livros do escritor
Roberto Freire, você poderá encontrar uma vasta abordagem de temas que
ampliam o entendimento e a compreensão da Soma. Mas são especialmente os
livros \emph{O Tesão pela Vida} (Ed. Francis) em que Freire e outros
somaterapeutas apresentam de maneira ampla e detalhada os principais
temas ligados à Soma. O livro \emph{Soma -- Vol 2} aborda especialmente sua
metodologia. Há também \emph{A Liberdade do Corpo} (Ed. Imaginário), livro
de João da Mata que aborda a utilização da capoeira angola como
instrumento terapêutico dentro dos grupos de Soma.

Roberto Freire construiu uma grande obra, algo em torno de trinta
livros, cujo tema central é a liberdade. Livros com \emph{Utopia e Paixão},
\emph{Sem Tesão não há Solução} e \emph{Ame e dê Vexame} além de terem sido
\emph{best"-seller} durantes vários anos, marcaram uma posição libertária
na cultura brasileira. Alguns de seus romances, como \emph{Cleo e Daniel} e
\emph{Coiote} seguem esta mesma perspectiva, expandindo a temática da
liberdade ao amor. Portanto, quem se interessa pela Soma e seus temas
correlatos, tem nos livros de Freire uma importante contribuição, além
de uma entrada ao pensamento libertário e ao prazer como virtude ética.

Como a Soma é constituída a partir de um campo transdisciplinar,
recomendamos a leitura de autores fundamentais para nós, nos campos da
Psicologia e da Filosofia, tais como: Wilhelm Reich, Friedrich Perls,
David Cooper, Alexander Lowen, Michel Foucault, Michel Onfray; e autores
libertários como: Max Stirner, Mikail Bakunin, Joseph Proudhon, Hakim
Bey e tantos outros. No Brasil, as pesquisas da Margareth Rago, Sílvio
Galo, Sérgio Norte, entre outros e do Núcleo de Sociabilidade Libertária
(Nu"-Sol), da \versal{PUC"-SP}, contempladas nos trabalhos de Edson Passetti,
Acácio Augusto, Salete Oliveira e Thiago Rodriguez.

É de quatro anos em média o tempo para a formação de um somaterapeuta.
Um ano de terapia, um ano como assistente de grupo e estudos teóricos e
um ano como co"-terapeuta de grupo e complementação teórica. No quarto
ano, ele terá o seu próprio grupo e trabalhará para sempre, fazendo a
sua própria terapia (individual) e recebendo supervisão de seu trabalho
por um somaterapeuta mais experiente.

\section*{Para obter mais informações sobre a Soma:}

Site:
\textless{}\emph{www.somaterapia.com.br}\textgreater{}

Canal da Soma no YouTube: somaterapia

E-mail: joaodamata@somaterapia.com.br